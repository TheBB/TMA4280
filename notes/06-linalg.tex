\chapter{Basic linear algebra performance}

\section{Introduction}

Simulation-based science and technology require a rich set of numerical
algorithms. For example, in the context of numerical solution of partial
differential equations we have seen the need to solve linear systems of
algebraic equations. Solution algorithms for linear system of equations may
again be classified as direct methods or iterative methods. In either case, the
solution algorithms rely on basic linear algebra operations. Most of the
floating point operations in a typical simulation code are associated with such
operations.

Because of the importance of basis linear algebra operations, a special library
called BLAS (\emph{Basic Linear Algebra Subroutines}) has been developed to deal
with such operations. The BLAS library is again classified into three levels:
\begin{itemize}
\item Level 1 operations: vector-vector operations;
\item Level 2 operations: matrix-vector operations;
\item Level 3 operations: matrix-matrix operations.
\end{itemize}
Two examples of level 1 BLAS operations are
\begin{align}
  \bm y := a \bm x + \bm y
  \label{eq:daxpy}
\end{align}
and
\begin{align}
  \sigma = \bm x \cdot \bm y = \bm x^\intercal \bm y
  \label{eq:dot}
\end{align}
Operation \eqref{eq:daxpy} is called a {\em daxpy} operation. Here, the input
$\bm x$ of length $n$ is scaled with a constant $a$ and added to the second
input vector $\bm y$, which is also of length $n$. The result is then stored
back in the vector $\bm y$, i.e. the original values in $\bm y$ are overwritten.

Operation \eqref{eq:dot} represents a dot product (or inner product) which we
have discussed extensively earlier in the course. Here, from the two input
vectors $\bm x$ and $\bm y$ (both of length $n$) we compute the scalar $\sigma$.

An example of a level 2 BLAS operation is the matrix-vector product
\begin{align}
  \bm y = \bm A \bm x
  \label{eq:mvp}
\end{align}
Here, the output vector $\bm y$ (of length $m$) is computed from the given input
matrix $\bm A$ (of dimension $m \times n$) and the given input vector $\bm x$
(of length $n$).

An example of a level 3 BLAS operation is the matrix-matrix product
\begin{align}
  \bm C = \bm A \bm B
  \label{eq:mxm}
\end{align}
Here, the matrix $\bm X$ (of dimension $m\times n$) is computed as the product
of the matrix $\bm A$ (of dimension $m\times k$) and the matrix $\bm B$ (of
dimension $k\times n$).

In this set of lecture notes, we will discuss the performance of operations
\ref{eq:daxpy}, \ref{eq:dot} and \ref{eq:mxm} on Vilje. We will primarily focus
on the single-processor performance, but we will also consider the
multi-processor performance using the BLAS implementation included in the MKL
library developed at Intel.

In particular, we will discuss the performance as a function of:
\begin{itemize}
\item basic linear algebra operation;
\item programming aspects;
\item high level programming languages;
\item exploiting multiple threads.
\end{itemize}

\subsection{Key data for Vilje}

The current supercomputer at NTNU, Vilje, is based on 1404 nodes. The total
number of processors (or cores) is 22464 physical cores, each core being
hyperthreaded, thus having 44928 logical cores.

Each node represents a shared memory system with 2 octave-core Sandy Bridge
chips which share 32 GB memory (although a few nodes have 128 GB memory).

Each processor operates at a clock rate of 2.6 GHz. The size of the private L1
cache is 32 kbyte for data and 32 kbyte for instructions. The size of the
private L2 cache is 256 kB, while the size of the off-chip L3 cache is 20 Mbyte.

The latency associated with the different memory levels is: 8 clock cycles for
L2, 30 clock cycles for L3, and 150 clock cycles for main memory.

\subsection{Maximum theoretical performance}

The maximum theoretical performance (or peak performance) is the maximum number
of floating point operations completed per second. We may talk about the maximum
theoretical performance for a single CPU, a single node, or for the entire
machine.

The maximum theoretical performance of a single core of a Sandy Bridge chip is
found as follows. First, each physical has a separate SIMD floating point unit
(AVX). This is a superscalar/FMA capable (Fused Multiply and Add) vector unit
which can operate on 4 double precision number simultaneously. This performance
is achieved if the units can be filled fast enough with data, and after a
certain start-up period (recall the earlier discussion about pipelining). With 1
AVX per physical core, the maximum theoretical performance per core is thus 8
floating point operations per clock cycle. With a clock cycle of 2.6 GHz, this
translates into 21 Gflops per physical core. Note, since logical cores share AVX
units, we cannot expect additional performance from using the hyperthreads,
since in order to achieve this performance we are already using the full memory
bandwidth of the machine.

Unfortunately, many operations only achieve a fraction of the maximum
theoretical performance. The measured performance will typically depend on the
the reuse of data (a high degree of reuse means less memory traffic) and how
good the compiler is. For the basic linear algebra operations we will study
here, we will see a large variation in performance. Not surprisingly, the
specially developed BLAS library will typically give excellent performance.
However, some of the observed differences may come as a surprise.

\section{Compiling and running the programs}

The source code and the buildsystem for the tests are found in the git
repository at \url{https://github.com/akva2/tma4280}. CMake is used to generate
the different Fortran programs, and to generate a build system capable of
building the testing suite on most machines, including Vilje, Kongull or your
local Linux/OSX machine. There is also a example job script (for Vilje, it needs
some small changes for Kongull), a script to postprocess the results and (for
those interested) the Octave/Matlab scripts used to generate the \LaTeX~code
for the tables in this document.

\section{Vector-vector operations}

In this section we briefly discuss some performance results for an important
vector-vector operation: the daxpy-operation \eqref{eq:daxpy}, which named as
such: \textbf{d}ouble precision \textbf{a}lpha $\bm x$ \textbf{p}lus $\bm y$.
\[
  \bm y := \alpha \bm x + \bm y,
\]
does exactly $n$ operations (additions) on $\mathcal{O}(n)$ data, and needs to
store $n$ floating point numbers back into memory. All this memory traffic will
severely influence the performance we can reach. In contrast to the
matrix-matrix multiplication, only $\mathcal{O}(1)$ floating point operations
are needed per floating point number stored. There is thus very little "reuse"
of data in vector-vector operations. From these general considerations we expect
vector-vector operations to perform significantly worse than the matrix-matrix
product multiplication.

\subsection{Performance results for daxpy}

\Autoref{tab:tab9, tab:tab10} show the performance results of the daxpy
operation. In general, a standard implementation of the daxpy operation in C
(i.e., a single loop), performs just fine. This is to be expected since this
operation is utterly memory bandwidth bound and not much can be done wrong.

\begin{table}
  \caption{
    Performance results (in MFlops) for a standard daxpy implementation in C
    (implemented as a single loop), compared with the performance using BLAS.
  }
  \label{tab:tab9}
  \begin{center}
  \bgroup\def\arraystretch{1.2}
  \begin{tabular}{c|rrrr|r}
    \hline
    $n$ & \texttt{-O0} & \texttt{-O1} & \texttt{-O2} & \texttt{-O3} & BLAS \\
    \hhline{======}
    $10^2$ & 45.92 & 63.69 & 82.82 & 68.37 & 0.10 \\
    $10^3$ & 227.13 & 363.00 & 457.12 & 434.61 & 0.81 \\
    $10^4$ & 403.55 & 1122.64 & 1217.60 & 1246.42 & 7.76 \\
    $10^5$ & 415.85 & 1300.58 & 1322.22 & 1338.98 & 79.71 \\
    $10^6$ & 379.52 & 418.88 & 413.65 & 413.65 & 341.52 \\
    $10^7$ & 376.78 & 420.55 & 413.65 & 420.36 & 341.52 \\
    \hline
  \end{tabular}
  \egroup
\end{center}

\end{table}

\begin{table}
  \caption{
    Performance results (in MFlops) for a standard daxpy implementation in
    Fortran (implemented as a single loop), compared with the performance using
    BLAS.
  }
  \label{tab:tab10}
  \begin{center}
  \bgroup\def\arraystretch{1.2}
  \begin{tabular}{c|rrrr|r}
    \hline
    $n$ & \texttt{-O0} & \texttt{-O1} & \texttt{-O2} & \texttt{-O3} & BLAS\\
    \hhline{======}
    $10^2$ & 0.07 & 0.07 & 0.08 & 0.07 & 0.10 \\
    $10^3$ & 0.89 & 0.83 & 0.67 & 0.69 & 0.80 \\
    $10^4$ & 9.37 & 7.23 & 8.17 & 7.30 & 7.75 \\
    $10^5$ & 70.96 & 66.18 & 70.65 & 87.33 & 79.70 \\
    $10^6$ & 353.27 & 356.28 & 385.78 & 402.52 & 341.51 \\
    $10^7$ & 513.54 & 514.07 & 514.52 & 512.66 & 341.51 \\
    \hline
  \end{tabular}
  \egroup
\end{center}

\end{table}

\section{Matrix-matrix multiplication}

Let us first consider the matrix-matrix multiplication \eqref{eq:mxm}. In
general, if $\bm A \in \mathbb{R}^{m \times k}$, $\bm B \in \mathbb{R}^{k \times
n}$ and $\bm C \in \mathbb{R}^{m \times n}$,
\[
  c_{ij} = \sum_{l=1}^k a_{il}b_{lj}, \qquad
  \forall\ i=1,\ldots,m, \qquad
  \forall\ j=1,\ldots,n.
\]
Written out, this is
\[
  c_{ij} = a_{i1}b_{1j} + a_{i2}b_{2j}+\cdots+a_{ik}b_{kj} \qquad
  \forall\ i=1,\ldots,m, \qquad
  \forall\ j=1,\ldots,n.
\]
Mathematically, there are thus $k$ multiplications and $k-1$ additions for each
index $i,j$. The total number of operations is then
\[
  \mathcal{N}_\text{op} = mn\left(2k-1\right).
\]

For the special case where $A,B,C\in \mathbb{R}^{nxn}$, the total number of operations is
\[
  \mathcal{N}_\text{op} = n^2²\left(2n-1\right) \approx 2n^3.
\]
Hence, if $m,n,k$ are of the same order (e.g., $m=k=n$),
\begin{align}
  \mathcal{N}_\text{op} = \mathcal{O}(n^3).
\end{align}

\subsection{Triple-nested loop}

In a numerical program, matrix multiplication can be implemented in several ways.
The most straightforward method is in a triple-nested loop as follows (using C).
\begin{lstlisting}[style=c]
  for(i=0; i<m; i++) {
    for(j=0; j<n; j++) {
      c[i][j] = 0.0;
      for(l=0; l<k; l++) {
        c[i][j] += a[i][l]*b[l][j];
      }
    }
  }
\end{lstlisting}
Here, the terms are accumulated relative to an initialized value of zero.
Counting the number of floating point operations in the inner-most loop, there
are one multiplication and one addition. The total number of floating point
operations for matrix multiplication then becomes $\mathcal{N}_\text{op} = 2mnk$
( $=2n^3$ when $m=k=n$).

\subsection{Loop unrolling}

Loop unrolling helps to make the data flow and the data dependencies more
explicit and prepare for good use of the floating point units. For example, when
$k=10$, the inner-most loop can be unrolled manually to get the following
alternative version:
\begin{lstlisting}[style=c]
  for (i=0; i<m; i++) {
    for (j=0; j<n; j++) {
      c[i][j] = a[i][0]*b[0][j]
              + a[i][1]*b[1][j]
              + a[i][2]*b[2][j]
              + a[i][3]*b[3][j]
              + a[i][4]*b[4][j]
              + a[i][5]*b[5][j]
              + a[i][6]*b[6][j]
              + a[i][7]*b[7][j]
              + a[i][8]*b[8][j]
              + a[i][9]*b[9][j];
    }
  }
\end{lstlisting}
Here, all the terms are written out explicitly, following the mathematical
definition. The computation of \texttt{c[i][j]} now takes one addition fewer.

Loop unrolling reveals independent operations that can be performed
concurrently, while reducing the index-related overhead of the loop. This makes
it possible to make better use of the pipelined, superscalar (vector) floating
point units of the processor. At high enough optimization levels, the compiler
will typically try to unroll loops where it sees this as beneficial.

\subsection{Performance results}

We now present performance measurements for the matrix-matrix multiplication.
The source codes used are given in the appendix. In all the tests we perform,
$m=k=n$ (i.e., we consider square matrices). Both a C version and a Fortran
version of the matrix-matrix operation will be tested and compared with the
performance using the \texttt{dgemm} routine from the BLAS library. We provide
the full program listings, together with a description of how the programs were
compiled and run on Vilje. For example, we will explore the effect of using
different levels of compiler optimization.

Note that we may call the Level 3 BLAS routine \texttt{dgemm} from either
Fortran or C. When we use this library routine, the optimization level or code
language should not matter. As long as $n$ is large enough, it is possible to
obtain a high degree of peak performance.

Finally, note that the performance results we list in the following are
approximate and may not be exactly reproducable. However, they represent typical
results and the conclusions we arrive at should be valid.

The programs were run on 16 processors even though the code contains no
communication between the processors. This gave 16 timing results per run.

These versions are single threaded; the difference is only the compiler
optimization level.

\begin{table}
  \caption{Standard mxm (triple-nested loop) using C, compared with BLAS.}
  \label{table:tab1}
  \begin{center}
  \bgroup\def\arraystretch{1.2}
  \begin{tabular}{c|rrrr|r}
    \hline
    $n$ & \texttt{-O0} & \texttt{-O1} & \texttt{-O2} & \texttt{-O3} & BLAS \\
    \hhline{======}
    100 & 274.55 & 566.00 & 562.76 & 563.32 & 775.64 \\
    500 & 226.53 & 503.96 & 496.40 & 496.84 & 14323.68 \\
    1000 & 180.78 & 214.33 & 214.37 & 214.36 & 16786.90 \\
    1500 & 169.95 & 205.26 & 204.69 & 204.55 & 17808.83 \\
    \hline
  \end{tabular}
  \egroup
\end{center}

\end{table}

As can be seen from \autoref{table:tab1}, the compiler optimization level had
really disappointing influence on the obtained performance---in fact, it
actually reduced it in some cases. These results looks far from flattering for
the C programming language. Fortunately, this can be overcome by using BLAS, in
this case through MKL. The BLAS version of the program was invoked by calling
e.g.

\begin{lstlisting}[style=shell]
  mpirun -np 16 timing-O3 1000 2
\end{lstlisting}

For the Fortran version of the programs, however, as can be seen from
\autoref{table:tab2}, the compiler optimization level greatly influences the
obtained performance.

\begin{table}
  \caption{
    Standard $m \times m$ (triple-nested loop) using Fortran, compared with
    BLAS.
  }
  \label{table:tab2}
  \begin{center}
  \bgroup\def\arraystretch{1.2}
  \begin{tabular}{c|rrrr|r}
    \hline
    $n$ & \texttt{-O0} & \texttt{-O1} & \texttt{-O2} & \texttt{-O3} & BLAS \\
    \hhline{======}
    100 & 247.37 & 1666.47 & 4800.47 & 7109.23 & 775.64 \\
    500 & 191.22 & 982.85 & 3375.70 & 7565.13 & 14323.68 \\
    1000 & 142.11 & 210.15 & 1356.46 & 7228.42 & 16786.90 \\
    1500 & 122.92 & 205.13 & 1357.79 & 7584.47 & 17808.83 \\
    \hline
  \end{tabular}
  \egroup
\end{center}

\end{table}

We see that the language utilized greatly influnces the performance of this
operation. This can be attributed to the fact that, given proper memory
handling, the matrix times matrix operation is dominated by floating point
operations. The BLAS and (and to some extent) Fortran realizations manage to
keep the pipelines filled with data, while the C version does not seem to be
able to keep the floating point units fully occupied.

Let us also compare these performance results with the manually unrolled
innermost loop (which we have done for the specific case $n=10$). The programs
are compiled and linked as before. When we activate \texttt{mxm\_unr}, we obtain
the results in \Autoref{table:convergence_C, table:convergence_F}.

\begin{table}
  \caption{``Manually'' unrolled innermost loop (\texttt{mxm\_unr}), C.}
  \label{table:convergence_C}
  \begin{center}
  \bgroup\def\arraystretch{1.2}
  \begin{tabular}{c|rrrr}
    \hline
    $n$ & \texttt{-O0} & \texttt{-O1} & \texttt{-O2} & \texttt{-O3} \\
    \hhline{=====}
    10 & 328.19 & 485.23 & 577.84 & 409.26 \\
    \hline
  \end{tabular}
  \egroup
\end{center}

\end{table}

\begin{table}
  \caption{``Manually'' unrolled innermost (\texttt{mxm\_unr}), Fortran.}
  \label{table:convergence_F}
  \begin{center}
  \bgroup\def\arraystretch{1.2}
  \begin{tabular}{c|rrrr}
    \hline
    $n$ & \texttt{-O0} & \texttt{-O1} & \texttt{-O2} & \texttt{-O3} \\
    \hhline{=====}
    10 & 189.17 & 292.61 & 674.69 & 621.29 \\
    \hline
  \end{tabular}
  \egroup
\end{center}

\end{table}

We do not observe much performance increase for either case. It seems that this is a trick
the compiler uses extensively on its own. Note that the Fortran version is still
about 50\% faster than the C version (for \texttt{-O3}).

We now try to use the parallel (multi-threaded version) of BLAS implemented in
the MKL library.
\begin{table}
  \caption{BLAS with 16 threads (SMP). All performance numbers are in MFlops.}
  \label{table:tab4}
  \begin{center}
    \bgroup\def\arraystretch{1.2}
    \begin{tabular}{cr}
      \hline
      $n$ & BLAS (16 threads) \\
      \hhline{==}
      100 & 1463.80 \\
      500 & 66599 \\
      1000 & 100527 \\
      1500 & 101389 \\
      \hline
    \end{tabular}
    \egroup
  \end{center}
\end{table}

From \autoref{table:tab4} we see that we obtain very impressive performance.
These numbers should be compared with the maximum theoretical performance for
the entire node (i.e., using 16 processors) which is $16\times 21$ GFlops $=
336$ GFlops. Note that the cannot get more than 21 GFlops per core since this
number corresponds to the maximum performance of the available multiply-and-add
units per core. We thus see that BLAS (using the SMP version of the \texttt{MKL}
library) achieves close to 33\% of the maximum theoretical performance over 16
processors. This is actually quite impressive since this was achieved only by
adding a few compile and run flags; the program itself is unchanged from
earlier. This might seem a bit disappointing, but it just shows the real
bottleneck in the computer: memory bandwidth.

Finally, we recall the reason for the excellent performance of the BLAS library
for the matrix-matrix multiplication operation: the mxm operation uses
$\mathcal{O}(n^2)$ data and $\mathcal{O}(n^3)$ operations, implying that we need
to do $\mathcal{O}(n)$ floating point operations per single floating point
number stored. Because of the significant "reuse" of data, there is a
significant potential to hide the memory latency and keep the floating point
units busy. In particular, BLAS is able to get close to optimal single-processor
performance, and a standard triple loop in Fortran is able to get fairly close
using two threads per core. Unfortunately, the C version is not able to keep the
floating point units busy enough.

\subsection{Combining Fortran and C}

We recall that memory is allocated column-wise in Fortran and row-wise in C. In
BLAS the Fortran convention is used. It is thus necessary to be careful when
using two-dimensional arrays when calint \texttt{dgemm} (or other routines
following the Fortran convention).

To ensure correctness, there are several ways to proceed. By switching indices
when accessing arrays in C, a ``pseduo'' column-wise allocation is achieved.
This is the approach we choosed to use here (and in the rest of the course).
Sometimes it may be possible to use a version of the library that accepts
row-wise allocation, such as the CBLAS library. However, CBLAS is not as
universally available, so we have chosen the approach that is most portable.

% The standard inner product (or dot product) of two vectors $x$ and $y$
% of length $n$ is given by
% \[
%         \sigma = \underline{x}^T\underline{y} = \underline{x}\cdot \underline{y} =
%         \sum_{i=1}^n x_i y_i.
% \]
% This operation requires $\mathcal{O}(n)$ floating point operations
% ($n$ multiplications and $n-1$ additions) on
% $2n = \mathcal{O}(n)$ floating point numbers, and only a
% single number ($\sigma$) needs to be stored back to memory.

% \begin{figure}
% \centering
%         \subfigure[Fortran]{\includegraphics[width=10cm]{daxpy-smp-fortran}}\\
%         \subfigure[BLAS]{\includegraphics[width=10cm]{daxpy-smp-blas}}
%         \caption{Performance of the daxpy operation in SMP mode. All programs are compiled with full compiler optimizations.}
%         \label{fig:daxpy-smp}
% \end{figure}
% \input{Table12.tex}

% \subsection{Performance results: inner product}

% Tables 7-10 show the performance of the inner product on \texttt{njord}.

% \begin{center}
  \bgroup\def\arraystretch{1.2}
  \begin{tabular}{c|rrrr|r}
    \hline
    $n$ & \texttt{-O0} & \texttt{-O1} & \texttt{-O2} & \texttt{-O3} & BLAS \\
    \hhline{======}
    $10^2$ & 45.92 & 63.69 & 82.82 & 68.37 & 0.10 \\
    $10^3$ & 227.13 & 363.00 & 457.12 & 434.61 & 0.81 \\
    $10^4$ & 403.55 & 1122.64 & 1217.60 & 1246.42 & 7.76 \\
    $10^5$ & 415.85 & 1300.58 & 1322.22 & 1338.98 & 79.71 \\
    $10^6$ & 379.52 & 418.88 & 413.65 & 413.65 & 341.52 \\
    $10^7$ & 376.78 & 420.55 & 413.65 & 420.36 & 341.52 \\
    \hline
  \end{tabular}
  \egroup
\end{center}

% \begin{center}
  \bgroup\def\arraystretch{1.2}
  \begin{tabular}{c|rrrr|r}
    \hline
    $n$ & \texttt{-O0} & \texttt{-O1} & \texttt{-O2} & \texttt{-O3} & BLAS\\
    \hhline{======}
    $10^2$ & 0.07 & 0.07 & 0.08 & 0.07 & 0.10 \\
    $10^3$ & 0.89 & 0.83 & 0.67 & 0.69 & 0.80 \\
    $10^4$ & 9.37 & 7.23 & 8.17 & 7.30 & 7.75 \\
    $10^5$ & 70.96 & 66.18 & 70.65 & 87.33 & 79.70 \\
    $10^6$ & 353.27 & 356.28 & 385.78 & 402.52 & 341.51 \\
    $10^7$ & 513.54 & 514.07 & 514.52 & 512.66 & 341.51 \\
    \hline
  \end{tabular}
  \egroup
\end{center}

% \begin{figure}
%         \includegraphics{inner}
%         \caption{Performance of the inner product operation in a single core,
%         single thread mode.
%         All programs are compiled with full compiler optimizations.
%         We now see that the specific programming language has less influence
%         on the performance compared to the matrix-matrix operation.
%         This can be attributed to the fact that a vector-vector operation
%         is latency bound and that the main aspect limiting the performance is
%         how fast the machine can feed the processor data.}
%         \label{fig:inner}
% \end{figure}

% \input{Table7.tex}
% \begin{figure}
%         \includegraphics{inner-smt}
%         \caption{Performance of the inner product operation in single core, SMT mode.
%         All programs are compiled with full compiler optimizations.
%         We now see that the specific programming language has less influence
%         on the performance compared to the matrix-matrix operation.
%         This can be attributed to the fact that a vector-vector operation
%         is latency bound and that the main aspect limiting the performance is
%         how fast the machine can feed the processor data.}
%         \label{fig:inner-smt}
% \end{figure}

% \input{Table8.tex}
% \begin{figure}
% \centering
%         \subfigure[FORTRAN]{\includegraphics[width=10cm]{inner-smp-fortran}}\\
%         \subfigure[BLAS]{\includegraphics[width=10cm]{inner-smp-blas}}
%         \caption{Performance of the inner product operation with $x \neq y$ in SMP mode. All programs are compiled with full compiler optimizations.}
%         \label{fig:inner-smp}
% \end{figure}


% \clearpage
% \subsection{Performance results: inner product (x=y)}

% Finally Tables 13-16 show the performance of an inner product
% when $x$ and $y$ is the same vector.

% \input{Table13.tex}
% \input{Table14.tex}
% \begin{figure}
%         \includegraphics{dot}
%         \caption{Performance of the inner product operation where $x = y$.
%                          All programs are compiled with full compiler optimizations.
%                          We see that this time the language utilized has less influnces
%                          on the performance of the operation. This can be attributed to the fact that this
%                          operation is latency bound and that the main thing limiting the performance is
%                          how fast the machine can feed the processor data.}
%         \label{fig:dot}
% \end{figure}
% \input{Table15.tex}
% \begin{figure}
%         \includegraphics{dot-smt}
%         \caption{Performance of the inner product operation where $x = y$ in SMT mode.
%                          All programs are compiled with full compiler optimizations.
%                          We see that this time the language utilized has less influnces
%                          on the performance of the operation. This can be attributed to the fact that this
%                          operation is latency bound and that the main thing limiting the performance is
%                          how fast the machine can feed the processor data.}
%         \label{fig:dot-smt}
% \end{figure}

% \input{Table16.tex}
% \begin{figure}
%         \subfigure[Fortran]{\includegraphics[width=6.5cm]{dot-smp-fortran}}
%         \subfigure[BLAS]{\includegraphics[width=6.5cm]{dot-smp-blas}}
%         \caption{Performance of the inner product operation where $x = y$ in SMP mode. All programs are compiled with full compiler optimizations.}
%         \label{fig:dot-smp}
% \end{figure}

% \end{document}

% \clearpage
% \section{Discussion}


% \subsection*{Memory limitations}
% In the vector-vector operation \emph{daxpy} a vector is multiplied with a scalar,
% and the result is added to another vector, overwriting the second vector. In Fortran
% language this will look like the following loop:
% \begin{lstlisting}[language=Fortran]
% real*8 a, x(n), y(n)
% do i = 1,n
%         y(i) = y(i) + a*x(i)
% enddo
% \end{lstlisting}
% Each iteration of the loop, the two elements $x(i)$ and $y(i)$ are loaded to registers,
% a multiply-add operation is performed on the data, and the result $y(i)$ is stored
% to memory. (In addition, some indexing work and a test for completion of the loop must be
% performed. The scalar constant a can be held in a register, without any need for extra
% memory operations.)

% The load/store unit can be used concurrently with the floating-point units.
% Each cycle, it can perform either a load or a store operation. At best,
% a store operatin in this example can be performed every third cycle, with two
% loads in between. Since the stored value is the result of one multiply-add operation,
% the maximum achievable performance for \emph {daxpy} is one third of peak performance.

% Figure \ref{fig:daxpy} demonstrates achievable performance of the daxpy operation on Njord.



% \subsection*{Data reuse in matrix multiplication}
% The key to higher performance than what was possible in the pervious example, is to perform
% more operations on the data that is being loaded, before storing the results.

% In matrix multiplication the amount of work is $\mathcal{O}\left(N^3\right)$, while the amount
% of data is $\mathcal{O}\left(N^2\right)$. Therefore, this problem has a potential for
% data resue. By rewriting the nested loop, the number of load and store operations
% can be reduce to the same as the nmber of multiply-add operations (or lower).

% The way this is accomplished is by performing outer loop unrolling. When this technique
% is used, a small block of values from one of the arrays is stored in registers for reuse.

% If both the outer loops are unrolled twice, the Fotran matrix multiplication
% loop would look similar to the following:
% \begin{lstlisting}
% do j = 1,n, 2
%         do l=1,k,2
%                 b00 = b(l,j)
%                 b01 = b(l,j+1)
%                 b01 = b(l+1,j)
%                 b11 = b(l+1,j+1)
%                 do i= 1,m
%                         c(i,j)   = c(i,j)   + a(i,1)  *b00
%                         c(i,j+1) = c(i,j+1) + a(i,1)  *b01
%                         c(i,j)   = c(i,j)   + a(i,1+1)*b10
%                         c(i,j+1) = c(i,j+1) + a(i,1+1)*b11
%                 enddo
%         enddo
% enddo
% \end{lstlisting}

% In the inner loop, there are now 6 memory operations (4 loads and 2 stores),
% whle there are 4 multiply-add operations. Four registers are needed to hold
% the constant array elements from the \emph{b} array. As the degree of unrolling
% increases, the number of memory operations can be reduced further. However,
% the number of available registers eventually becomes a bottleneck.

% The SGI compiler system has a Loop Nest Optimizer (LNO) module that
% performs outer loop unrolling at \emph{-O3} and higher, in an early
% compilation stage. The inner loop is unrolled if necessary in the final stages
% of the compilation, using so-called software pipelining principles.

% \subsection*{Cache optimizations}
% The outer loop unrolling enables the processor to reach performance levels
% close to the theoretical maximum, by converting the loop from memory
% bound to floating point bound. However, when the matrix dimension increases,
% performance will fall as lower levels of the memory hierarchy are accessed.

% The LNO module will use other forms of optimizations to improve cache and
% memory behavior. Loop interchange may enabled stride-1 acess to arrays.
% If the data structures are too big to fin in (usually) L2 cache, cache blocking
% is performed. The matrices or arrays are split into smaller pieces or blocks
% that fits in cache. The matrix multiplication is then performed in a series
% of sub-matrix multiplies.

% Another type of optimization the LNO module performs is prefetching. Data
% can be moved from meory into cache befroe use, hiding the meory acess
% time. In a similar way load operations can be performed as early as possible
% to hide the load latency from L1 cache.

% The combined effect of the optimizations performed by the LNO module can
% greatly enhance performance. However, it can be hard to understand the transformed
% version of a loop like the matrix multiplication.

% \subsection*{Measured performance of matrix multiplication}
% The most obvious factor in performance is the matrix dimension. For $n=10$,
% high overhead costs can make performance sub-optimal. For $n=100$, and
% particular $n=500$, a good version of the code will be able to achieve
% high performance, as the data fits in L2 cache.

% As the matrix dimensions approaches 600, there is an increasing pressure
% on L2 cache. At $n=1000$, the performance will to a large degree be a measure
% of how effective the cache optimization techniques have been.

% Except for the smallest problem size, \emph{dgemm} delivers high performance.
% The BLAS library is optimized for problems sizes that needs good use of the lower
% levels of the memory herarchy. BLAS exists for all platforms used in scientific
% computing, and is often tuned by the system vendor using automated optimization techniques.

% For smaller problem sizes, the best performance may be achieve by the
% compiler, by carefully writing an unrolled or cache blocked version, or by a
% combination of these. The Fortran compiler was able to produce code that
% gave high performance for a range of problems sizes. The manually unrolled versions
% had decent performance as well.

% \subsection*{Language differences}
% It is apparent that the Fortran version of the code delivers much higher
% performance than the C version. The reason is inherent difficulties with optimization
% of C code. While performing optimizations, the particular optimal scheduling of
% instructions in an inner loop, the compiler bust be certain that the correctness of results
% are not effected.

% The use of pointers in C introduces a difficulty for the compiler. Two pointer
% variables may in principle refer to identical memory locations. There is no way
% for the compiler to know whether such aliasing does in fact occur. It is possible
% to specify that pointers do not point to overlapping regions of memory.

% For the \emph{daxpy} operation, this made it possible to achieve equal performance
% levels under C and Fortran. For the matrix multiplication, the C version showed
% improved performance using this compiler option, but the results
% were still not as good as under Fortran. This probably had to do with the use of
% two-dimentional arrays.

% \subsection*{Pitfalls when combining Fortran and C}
% Memory is allocated column-wise in Fortran and row-wise in C. In the BLAS library
% the Fortran convention is used. It is necessary to be careful when using two-dimensional
% arrays when call \emph{dgemm} (or other routines following the Fortran convention).

% To ensure correctness, there are several ways to proceed. By switching indices
% when accessing arrays in C, a ``pseduo'' column-wise allocation is achieved.
% Sometimes it may be possible to use a version of the library that accepts row-wise
% allocation, such as the CBLAS library.

% It is also possible to rewrite the original matrix computation using transposed terms.
% This was done in the C version here. Since \emph{dgemm} can handle transposed arrays,
% in reality we used dgemm to compute $C^T = A^T\cdot B^T$.

% \subsection*{Conclusion}
% The matrix multiplication operation is an interesting problem in numerical programming.
% In principle, it may deliver high performance, by exploiting
% the potential of data reuse. However, this will not happen automatically. An
% implementation that is performing well fro some problem sizes, may not be appropriate
% for others.

% One option is to use the \emph{dgemm} routing from the BLAS Level 3 library. This may in
% particular be a good choice for large matrices. Figure \ref{fig:mxm} demonstrates
% that the compiler may produce fast code for a range of matrix dimensions that is L2 cache friendly.
% However, this may be language dependent.

% The compiler was able to produce very fast code from the simple Fortran triple-nested loop.
% Often bloated code is slow code. However, for the smallest problem sizes, it is possible
% to demonstrate that a carefully manually unrolled version delivers even better performance,
% even though it is using many lines of code (This is not shown here.) Performance measurements
% can help decide what will be the right way to implement a routine such as matrix multiplication.

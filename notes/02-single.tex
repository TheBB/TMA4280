\chapter{Single processor systems}

\section{A prototypical processor}

We start by explaining some of the basic tasks performed by a single processor.
The comments are particularly relevant for the MIPS R14000 processor used in a
previous supercomputer at NTNU (called \emph{Gridur}, a SGI Origin 3000 system
used in the period 2001--2007). The MIPS processor is an example of a processor
implementing a RISC architecture (RISC---\emph{Reduced Instruction Set
Computer}), which has been a very important processor design over the past
couple of decades. We will later return to comment on the POWER5 processor
used in the previous supercomputer at NTNU, in particular some of the key
differences compared to the processor discussed in this section. The name POWER
refers to \emph{Performance Optimization With Enhanced RISC}, and is also the
name of a series of microprocessors designed by IBM.

\begin{figure}[htbp]
  \centering
  \begin{tikzpicture}[
  block/.style={
    draw=darkblue,
    fill=cadet,
    shape=rectangle,
    rounded corners=1mm,
    text height=1.5ex,
    text depth=.25ex,
  },
  l1/.style={
    minimum height=8mm,
    minimum width=4cm,
  },
  ops/.style={
    minimum height=8mm,
    minimum width=4cm,
    align=right,
  },
  inst/.style={
    minimum height=28mm,
    minimum width=16mm,
  },
  line/.style={
    thick,
    draw=darkblue,
  }]
  \node[block, l1] (L1inst) {L1 Instruction Cache};
  \node[block, l1, right=15mm of L1inst] (L1data) {L1 Data Cache};
  \node[block, l1, above=8mm of L1inst, fill=salmon] (Rest) {RAM, disk, network};
  \node[block, l1, above=8mm of L1data, fill=salmon] (L2) {L2 cache};
  \node[block, ops, below=16mm of L1data.east, anchor=east] (LS) {Load and store};
  \node[block, ops, below=10mm of LS.east, anchor=east] (Int) {Integer};
  \node[block, ops, below=10mm of Int.east, anchor=east] (Float) {Floating point};
  \node[block, inst, below=26mm of L1inst.east, anchor=east] (Decode) {Decode};
  \node[block, inst, below=26mm of L1inst.west, anchor=west] (Branch) {Branch};
  \node[block, inst, below=2mm of Decode, minimum height=8mm] (Clock) {Clock};

  \draw[line, <->] (Rest.east) -- (L2.west);
  \draw[line, <->] (L1inst.east) -- (L1data.west);
  \draw[line, <->] (L1data.south) -- (LS.north);
  \draw[line, ->] (Decode.east) -- (Int.west);
  \draw[line, ->] ([yshift=10mm]Decode.east) -- (LS.west);
  \draw[line, ->] ([yshift=-10mm]Decode.east) -- (Float.west);
  \draw[line, ->] (Decode.west) -- (Branch.east);
  \draw[line, ->] ([xshift=12mm]L1inst.south) -- (Decode.north);
  \draw[line, <-] ([xshift=-12mm]L1inst.south) -- (Branch.north);
  \draw[line] ($(L1inst.east)!0.5!(L1data.west)$) -- ($(Rest.east)!0.5!(L2.west)$);
\end{tikzpicture}

  \caption{
    A prototypical processor, including the MIPS R14000 used in \emph{Gridur},
    the supercomputer at NTNU during the period 2001--2007. The red components
    are off-chip.
  }
  \label{fig:Lande}
\end{figure}

In order to introduce some key concepts in the RISC architecture, let us briefly
explain what happens when we perform the following simple operation
\begin{align}
  c = a + b.
  \label{single:add1}
\end{align}
Here, we want the processor to add the two numbers $a$ and $b$ and store the
answer in $c$. In this case, $a$ and $b$ are referred to as \emph{operands},
while $+$ is the \emph{operator}. Hence, the simple addition of two scalars
implies a single floating point operation.

The basic unit of ``time'' for our processor is a clock cycle. The state of the
processor changes from clock cycle to clock cycle, depending on what needs to be
done. Different parts/units of the processor work on different tasks
independently of each other. These tasks may correspond to different
instructions, each in a particular phase of completion.

Consider again the addition of two scalars. The operation \eqref{single:add1}
may be part of a larger program involving many operations (or instructions). Let
us assume that the instruction for the ``add'' operation is currently in a small
memory module denoted as L1 cache (for instructions). When the execution of our
program is ready to perform this operation, the instruction is brought into a
decoder and made ready for execution. The addresses of the operands ($a$ and
$b$) are computed and the operands are brought into two registers of the
processors. We assume that the operands are available in the small memory module
called L1 cache (for data).

The operands are then directed from the two registers to the floating point unit
for addition (denoted as FPAdd) where the two numbers are added together. The
whole operation takes just a few clock cycles. For the MIPS R14000 processor,
this operation takes five clock cycles:
\begin{enumerate}
\item read from register
\item align
\item add
\item pack
\item write to register
\end{enumerate}
Note that the number of clock cycles required to perform this operation is not
the same as the number of bits used to store our operands (e.g., 64 bits in
double precision). The reason is that the data flow in the processor happens
along a ``wide bus'' capable of moving all the bits at the same time. This is an
example of bit-level parallelism. Older processors could only move 4, 8, 16, and
32 bits at a time and an add operation therefore took additional clock cycles to
complete. In addition, current processors have a much higher clock frequency (or
shorter clock period) compared to older processors.

Let us now make a few more remarks regarding the processor in
\autoref{fig:Lande}. We have already mentioned the floating point unit for
addition, FPAdd. However, the processor has also other functional units. For
example, the R14000 prosessor has five functional units which can operate
independently from each other:
\begin{itemize}
\item Load/Store: computes memory addresses and to bring operands to and from
the memory;
\item ALU1: a unit for addition and subtraction of integers, and logical
operations;
\item ALU2: a unit for addition, subtraction, multiplication, and division of
integers, as well as logical operations;
\item FPAdd: adds of floating point numbers;
\item FPMult: a unit for multiplication, division, and square root of floating
point numbers.
\end{itemize}
Note that the operands used in these functional units need to come from the
registers in the processor. If the operands are not in the registers, they have
to be brought from the memory into the registers with a {\em load} operation.
The answer (or output) from the functional units are also stored in registers
and subsequently stored in the memory with a {\em store} operation.

\section{Memory hierarchy}

The example from the previous section involving the addition of two floating
point numbers assumed that the instruction for the operation \eqref{single:add1}
was in the memory module denoted as L1 cache for instructions; see
\autoref{fig:Lande}. Similarly, we assumed that the operands $a$ and $b$ were
available in the memory module denoted as L1 cache for data.

Let us now comment a bit more on what happens if these assumptions are not true.
In this case, the program will check whether the instruction or data are
available in the memory module denoted as L2 cache in Figure \ref{fig:Lande}.
This is a memory module which is larger than L1 cache. Furthermore, the L2 cache
is not split into a separate module for instructions and a separate module for
data. It also takes longer (meaning more clock cycles) to fetch data from L2
cache compared to L1 cache. It could also happen that the instruction and the
data the program is looking for is not available in L2 cache either. In this
case data has to be brought in from main memory (RAM) or perhaps even further
away (e.g. the local disk).

Bringing instructions and data to and from memory (from cache, RAM, etc.) is
typically a bottleneck in scientific computing. The processors are getting
faster and faster, but the memory bandwidth (or transfer rate in bytes per
second) is not keeping up at a similar speed. Hence, for certain operations, the
processor can become ``starved'' for data, meaning that it is idle for much of
the time while waiting for data to be transferred to and from memory. The
overall performance in terms of floating point operations per second will in
such cases not be governed by the processor speed, but by the memory access
time.

In order to ``hide'' this difference in speed (i.e., processor speed versus
memory access speed), the memory is organized in a hierarchical fashion;
see \autoref{fig:MemoryHierarchy}.
\begin{figure}
  \centering
  \begin{tikzpicture}[
  line/.style={
    thick,
    draw=darkblue,
  }]
  \coordinate (A) at (-4.5,0) {};
  \coordinate (B) at (4.5,0) {};
  \coordinate (C) at (0,7.4) {};
  \fill[cadet] (A) -- (B) -- (C);
  \draw[line, name path=AC] (A) -- (C);
  \draw[line, name path=BC] (B) -- (C);
  \foreach \y/\A in
  {0/Tape, 1/Distributed memory, 2/Local disk, 3/Main memory (RAM), 4/Cache, 5/Registers, 6/CPU}
  {
    \path[name path=horiz] (A|-0,\y) -- (B|-0,\y);
    \draw[line, name intersections={of=AC and horiz,by=P},
    name intersections={of=BC and horiz,by=Q}] (P) -- (Q)
    node[midway, above=1mm, text height=1.5ex, text depth=.25ex] {\A};
  }
\end{tikzpicture}

  \caption{
    The memory hierarchy of a computer system. Higher entries are faster, while
    lower entries are cheaper and larger.
  }
  \label{fig:MemoryHierarchy}
\end{figure}
The fastest part of the memory is closest to the CPU. For example, on the MIPS
R14000, the L1 cache is on the chip. This part of the memory is fast, but also
small because it is very expensive. In contrast, the main memory is much larger,
but has also a much longer access time; see \autoref{tbl:cycles}. We will later
discuss in more detail how it is decided what should be in the L1 and L2 cache.

\begin{table}
  \centering
  \caption{
    Typical memory access times for R14000. The numbers represent number of
    clock cycles.
  }
  \label{tbl:cycles}
  \bgroup\def\arraystretch{1.2}
\begin{tabular}{rl}
  \hline
  Memory type & Clock cycles \\ \hhline{==}
  Registers & $1$ \\ \hline
  L1 cache & $2$--$3$ \\ \hline
  L2 cache & $10$--$12$ \\ \hline
  Main memory &  $100$--$200$ \\ \hline
  Message passing & $\mathcal{O}(10^3)$--$\mathcal{O}(10^4)$ \\ \hline
  Local disk & $\mathcal{O}(10^6)$ \\ \hline
\end{tabular}
\egroup

\end{table}

We remark that message passing in \autoref{tbl:cycles} refers to communication
between individual processors by sending messages over a network; we will return
to a discussion of multiple processors later.

\section{Pipelining/vectorization}

Assume now that, instead of (\ref{single:add1}), we would like to add the two
vectors $\bm a$ and $\bm b$ of length $n$ and store the result in a vector $\bm
c$, i.e.
\begin{align}
  \bm c = \bm a + \bm b.
  \label{single:addn}
\end{align}
We can also write \eqref{single:addn} as the loop (in Fortran)
\begin{lstlisting}[style=fortran]
  for i=1,n
    c(i) = a(i) + b(i)
  end
\end{lstlisting}
Again, we start by assuming that the data are available in the registers. From
there, the individual vector elements $a(i)$ and $b(i)$ enter the floating point
unit FPAdd where the numbers are added to produce the output elements $c(i)$,
$i=1,\ldots,n$.

We mentioned earlier that it takes five clock cycles to add two floating point
numbers together. This would perhaps suggest that the total number of clock
cycles for the operation \eqref{single:addn} is $5n$, and that the total
execution time therefore is $5n\tau$ where $\tau$ is the clock period. However,
if things are done optimally, the total number of clock cycles can be reduced to
approximately $n$ for $n\gg 1$. The reason for this is that the five stages in
the adder correspond to independent tasks. This means that, as soon as the
addition of two operands (i.e. $a(i)$ and $b(i)$) has finished the first stage,
two new operands ($a(i+1)$ and $b(i+1)$) can enter the first stage in the adder.
Hence, after five clock cycles, the first number $c(1)$ in the operation
\eqref{single:addn} is ready, while the numbers $c(2)$, $c(3)$, $c(4)$, $c(5)$,
and $c(6)$ are in different phases of completion in the adder.

In summary, after a few clock cycles to ``fill up'' the adder, a new answer
$c(i), i=2,\ldots,n$ is ready every clock cycle. This is what is referred to as
\emph{pipelining} (or sometimes \emph{vectorization}). The reason behind this
term is quite obvious: we constantly feed the floating point unit (in this case,
the adder) so that the ``pipeline'' is always full. In other words, we do not
wait until one answer is ready before we start the process of adding two new
numbers. In this way, we achieve a certain level of parallelism in the sense
that asymptotically (for long vector lengths), the adder works simultaneously on
five different pairs of operands.

The above discussion assumed that the data (i.e. the operands $a(i)$ and $b(i)$,
$i=1,\ldots,n$) are ready for the adder with no delay. Whether this is possible
or not depends on the particular processor. In the case of the MIPS R14000
processor it is not possible to achieve this performance. The reason is that,
in the best case, only a single floating point number can be brought between the
memory (L1 cache) and a register at a time. Since we need to fetch two operands,
$a(i)$ and $b(i)$, per floating point operation, and store the answer $c(i)$
back to memory, a minimum of three clock cycles are needed for memory transfer
per addition. Hence, even though the floating point unit (the adder) can
theoretically complete one addition per clock cycle, the memory traffic will be
the bottleneck, at least for large $n$.

The only possibility for achieving a better performance is if all the operands
are already available in the processors registers. However, since a processor
only has a limited number of registers (on the MIPS R14000 the number of
registers is 64), such performance cannot be acheived if $n \gg 1$.

Let us now predict the optimal performance for the operation \eqref{single:addn}
in the case $n \gg 1$ on Gridur. The clock cycle for each processor is 500 MHz.
Hence, the clock period is 2 ns. Since each floating point operation will
asymptotically require three clock cycles due to the fetch and store operations
(see the discussion above), each floating point operation will at least require
three clock cycles, or 6 ns. Hence, the maximum performance for the operation
\eqref{single:addn} is $(6\cdot 10^{-9})^{-1}$ floating point operations per
second, or approximately 167 MFlops. In practice, less performance may be
achieved, in particular if the operands need to first be brought in from deeper
layers of the memory hierarchy; see \autoref{fig:MemoryHierarchy}.

\section{Superscalar operations}

Consider now a modification of the operation \eqref{single:addn} to the
following operation:
\begin{align}
  \bm c = \bm a + \gamma \bm b.
  \label{single:addmultn}
\end{align}
Here, each vector element $b(i)$, $i=1,\ldots,n$ is multiplied with a scalar,
$\gamma$, before being added to $a(i)$. Similar to the operation
\eqref{single:addn} the result of each addition is stored as the vector element
$c(i)$.

The new operation here is the multiplication. As mentioned earlier, each R14000
processor has a separate floating point unit for multiplication, FPMult. Similar
to the add operation, multiplying two numbers also take five clock cycles:
\begin{enumerate}
\item read from register
\item multiply
\item sum product
\item pack
\item write to register
\end{enumerate}
Hence, all the comments made in the previous section for the operation
\eqref{single:addn} also apply if the add operation $+$ is replaced by
multiplication $\times$.

Let us now comment on what happens when we combine both multiplication and
addition as in \eqref{single:addmultn}. Again, let us first assume that all the
data are readily available (i.e. stored in the registers). The vector elements
$b(i)$ are brought to the multiplier where each element is multiplied by the
scalar $\gamma$; see Figure \autoref{fig:SuperScalar}. After a startup time of
five clock cycles, a new answer is coming out from the multiplier every clock
cycle. We assume here that the pipelining feature is exploited.

\begin{figure}[htbp]
  \centering
  \begin{tikzpicture}[
  block/.style={
    draw=darkblue,
    fill=cadet,
    shape=rectangle,
    rounded corners=1mm,
    minimum height=12mm,
    minimum width=12mm,
  },
  pinin/.style={pin edge={to-, thick, darkblue}},
  pinout/.style={pin edge={-to, thick, darkblue}},
  ]
  \node[block, pin={[pinin]above:$\gamma$}, pin={[pinin]left:$b(i)$}] (mult) {$\times$};
  \node[block, right=20mm of mult, pin={[pinin]above:$a(i)$}, pin={[pinout]right:$c(i)$}] (add) {$+$};
  \draw[darkblue, ->, thick] (mult.east) -- (add.west);
\end{tikzpicture}

  \caption{The superscalar operation multiply and add.}
  \label{fig:SuperScalar}
\end{figure}

Each output from the multiplier is now channeled directly as an input to the
adder where it is added to the vector element $a(i)$. After another five clock
cycles, the answer $c(i)$ is ready. Hence, after a startup time of 10 clock
cycles (5 for the multiplier and 5 for the adder), we get one complete answer
$c(i), \, i=1,\ldots,n$ as output every clock cycle. Asymptotically (i.e., for
large vector lengths), the theoretical performance is therefore two floating
point operations per clock cycle. This way of piping the output from one
floating point unit into the input for another unit is denoted as
\emph{superscalar} capability. Similar to the pipelining feature of the adder
and the multiplier, the superscalar capability offers yet another possibility of
parallelism in the sense that each single processor is capable of performing
addition and multiplication at the same time (for sufficiently long vector
lengths).

In practice, the processor has only a limited number of registers, and we need
to fetch the operands from memory (L1 cache) and store the answers in memory.
Similar to the operation \eqref{single:addn}, each complete vector element
$c(i)$ will require three clock cycles due to the memory traffic. However, in
contrast to the operation \eqref{single:addn}, the operation
\eqref{single:addmultn} implies two floating point operations instead of one for
each complete vector element $c(i)$. The maximum single-process performance we
can obtain on R14000 for \eqref{single:addmultn} is thus twice the performance
for \eqref{single:addn}, i.e. 333 MFLOPS.

\section{Cache}

Let us now discuss in more detail the interaction between the cache and the main
memory. The main purpose of the cache is to keep copies of data in extra (and
fast) memory close to the CPU in order to ``hide'' the relatively slow transfer
rate between the main memory and the processor.

Because fast caches are expensive, they tend to be small. As an example, we give
the memory sizes for the R14000 processors. On Gridur, four individual
processors shared up to 4 Gbytes of main memory. Each processor had an L2 cache
of size 8 Mbytes and two L1 caches (one for instructions and one for data), each
only of size 32 Kbytes. Hence, the L1 cache for data can only hold up to 4000
floating point numbers (assuming double precision), which is relatively small in
the context of simulating systems with thousands or millions of unknowns (e.g.,
for the numerical solution of partial differential equations). The L2 cache can
hold more data, but the transfer rate is a little bit longer compared to the L1
cache; see \autoref{tbl:cycles}.

The cache is smaller than the main memory by some power of two. Hence, a
strategy for mapping memory locations to cache locations needs to be defined. We
describe three strategies for doing this.

\subsection{Direct mapped cache}

One strategy is to use what is refered to as a \emph{direct mapped cache}. In
this case, each location in main memory corresponds to a unique location in
cache; see \autoref{fig:DirectMappedCache}. The main memory address is split
into two parts: the first bits of the memory address are called the \emph{set
bits} and these bits give the precise cache address. The remaining bits are
called the \emph{tag bigs}, and these are used to determine if a copy of the
content at the particular main memory location has been copied into the cache
location given by the set bits. With this strategy we see that several main
memory addresses map to the same cache address.
\[
  \text{Memory address} =
  \underbrace{b_1 \ldots b_k}_{\text{tag bits}}
  \underbrace{b_{k+1} \ldots b_{N}}_{\text{cache address}}.
\]

\begin{figure}
  \centering
  \begin{tikzpicture}[scale=0.3]
  \def \n {7};
  \def \k {3};
  \def \q {1};
  \def \s {3};

  \foreach \i [
  evaluate=\i as \R using -(\n+1)*\i,
  evaluate=\i as \L using -(\n+1)*(\i+1),
  ] in {0,...,\k}
  {
    \fill[cadet] (\R-\q, 0) -- (\R-\q, -2) -- (\R-\q-1, -2) -- (\R-\q-1, 0) -- (\R-\q, 0);
    \fill[salmon] (\R-\s, 0) -- (\R-\s, -2) -- (\R-\s-1, -2) -- (\R-\s-1, 0) -- (\R-\s, 0);
    \draw[darkblue, very thick] (\R, 0) -- (\R, -2) -- (\L, -2) -- (\L, 0) -- (\R, 0);
    \foreach \j [
    evaluate=\j as \l using \R-\j
    ] in {1,...,\n}
    {
      \draw[darkblue, very thin] (\l, 0) -- (\l, -2);
    }
  }

  \fill[cadet] (-\q, 3) -- (-\q, 1) -- (-\q-1, 1) -- (-\q-1, 3) -- (-\q, 3);
  \fill[salmon] (-\s, 3) -- (-\s, 1) -- (-\s-1, 1) -- (-\s-1, 3) -- (-\s, 3);
  \draw[darkblue, very thick] (0, 3) -- (0, 1) -- (-\n-1, 1) -- (-\n-1, 3) -- (0, 3);
  \foreach \j in {1,...,\n} {
    \draw[darkblue, very thin] (0-\j, 3) -- (0-\j, 1);
  }

  \node[anchor=west] at (0, -1) {RAM};
  \node[anchor=west] at (0, 2) {Cache};
\end{tikzpicture}

  \caption{
    A direct mapped cache. Each main memory address maps to a unique and
    pre-determined location in the cache.
  }
  \label{fig:DirectMappedCache}
\end{figure}

When some particular data is requested by the program, e.g., a floating point
number, the processor will check whether the data is stored in L1 cache. It does
this by looking up the cache address (taking the least significant bits of the
memory address) and checking whether the tag bits at that location match the tag
bits of the memory address. If is not in L1 cache, the processor will check
whether the data is stored in L2 cache. If this is the case, the requested data
will be copied from the L2 cache into L1 cache. If the data is not in L2 cache
either, the data will have to be brought in from main memory. In this case, a
copy will be made in L2 cache as well as in L1 cache. In either case, the tag
bits at the given cache address will be updated to match the new mapped
location.

Note that when a floating point number (or an integer) is requested by the
program, more than a single number is copied into cache. The minimum amount of
data copied is called a {\em cache line}. For the R14000 processor, the cache
line for the L1 cache is 32 bytes (corresponding to four floating point numbers
in double precision), while the cache line for the L2 cache is 128 bytes
(corresponding to 16 floating point numbers in double precision). The extra
numbers copied are the numbers in the adjacent memory locations in main memory.

Consider again the operation \eqref{single:addn}. Assume that the vectors $\bm
a$, $\bm b$ and $\bm c$ represent floating point numbers in double precision,
and that the vectors are stored after each other in main memory. Let the vector
length $n=4000$, i.e. each vector will precisely fill the L1 data cache. For
every element $c(i)$ computed, the operands $a(i)$ and $b(i)$ will need to be
brought in all the way from main memory due to the fact that $a(i)$, $b(i)$ and
$c(i)$ happen to have the same cache address. A severe drop in performance will
be observed in this case. This situation is refered to as cache trashing; see
\autoref{fig:CacheTrashing}.

\begin{figure}
  \centering
  \begin{tikzpicture}[scale=0.3]
  \def \n {7};
  \def \k {3};
  \def \q {1};

  \foreach \i [
  evaluate=\i as \R using -(\n+1)*\i,
  evaluate=\i as \L using -(\n+1)*(\i+1),
  ] in {0,...,\k}
  {
    \fill[cadet] (\R-\q, 0) -- (\R-\q, -2) -- (\R-\q-1, -2) -- (\R-\q-1, 0) -- (\R-\q, 0);
    \draw[darkblue, very thick] (\R, 0) -- (\R, -2) -- (\L, -2) -- (\L, 0) -- (\R, 0);
    \foreach \j [
    evaluate=\j as \l using \R-\j
    ] in {1,...,\n}
    {
      \draw[darkblue, very thin] (\l, 0) -- (\l, -2);
    }
  }

  \fill[cadet] (-\q, 3) -- (-\q, 1) -- (-\q-1, 1) -- (-\q-1, 3) -- (-\q, 3);
  \draw[darkblue, very thick] (0, 3) -- (0, 1) -- (-\n-1, 1) -- (-\n-1, 3) -- (0, 3);
  \foreach \j in {1,...,\n} {
    \draw[darkblue, very thin] (0-\j, 3) -- (0-\j, 1);
  }

  \node[anchor=west] at (0, -1) {RAM};
  \node[anchor=west] at (0, 2) {Cache};
  \node[anchor=north] at (-\q-0.5, -2) {\scriptsize $a(i)$};
  \node[anchor=north] at (-\n-1-\q-0.5, -2) {\scriptsize $b(i)$};
  \node[anchor=north] at (-\n-\n-2-\q-0.5, -2) {\scriptsize $c(i)$};
  \node[anchor=south] at (-\q-0.5, 3) {\scriptsize ??};
\end{tikzpicture}

  \caption{
    Cache trashing. The corresponding elements in the vectors $\bm a$, $\bm b$
    and $\bm c$ all map to the same cache address.
  }
  \label{fig:CacheTrashing}
\end{figure}

Cache trashing can be avoided by storing the elements in the vectors $\bm a$,
$\bm b$ and $\bm c$ in a different way; see Figure \ref{fig:AdjacentMemory}. We
remark that the crash trashing example given here is perhaps not likely to
happen. Nonetheless, it illustrates the point that severe performance
degradation is possible to observe due to undesirable memory traffic.

\begin{figure}
  \centering
  \begin{tikzpicture}[
  scale=0.8,
  every node/.style={
    text height=1.5ex,
    text depth=.25ex,
  }
  ]
  \draw[darkblue, very thick] (-6, 0) -- (6, 0);
  \draw[darkblue, very thick] (-6, 1) -- (6, 1);
  \draw[darkblue, very thin] (-3, 0) -- (-3, 1);
  \draw[darkblue, very thin] (-2, 0) -- (-2, 1);
  \draw[darkblue, very thin] (-1, 0) -- (-1, 1);
  \draw[darkblue, very thin] (0, 0) -- (0, 1);
  \draw[darkblue, very thin] (1, 0) -- (1, 1);
  \draw[darkblue, very thin] (2, 0) -- (2, 1);
  \draw[darkblue, very thin] (3, 0) -- (3, 1);
  \node at (-3.5, 0.5) {\footnotesize $\cdots$};
  \node at (-2.5, 0.5) {\footnotesize $a_i$};
  \node at (-1.5, 0.5) {\footnotesize $b_i$};
  \node at (-0.5, 0.5) {\footnotesize $c_i$};
  \node at (0.5, 0.5) {\footnotesize $a_{i+1}$};
  \node at (1.5, 0.5) {\footnotesize $b_{i+1}$};
  \node at (2.5, 0.5) {\footnotesize $c_{i+1}$};
  \node at (3.5, 0.5) {\footnotesize $\cdots$};
\end{tikzpicture}

  \caption{Adjacent memory layout.}
  \label{fig:AdjacentMemory}
\end{figure}

\subsection{Fully associative cache}

To avoid the possibility of cache trashing, we can use a different cache
strategy called a \emph{fully associative cache}. In a fully associative cache,
each cache address can map to any memory address. This is essentially a direct
mapped cache where all the bits are used as tag bits and no bits are used as a
cache address.

To find the corresponding cache location for a given memory address, the tag
bits of all cache locations must be checked. This is typically done in parallel
using dedicated hardware, which is rather complicated. For this reason, fully
associative caches are rarely seen.

When the cache is full, a cache line must be evicted to make way for new data.
How this happens depends on the replacement policy:
\begin{itemize}
\item Least recently used (LRU);
\item Least frequently used (LFU);
\item Random
\end{itemize}
In the context of numerical solution of partial differential equations, the
alternative \emph{LRU} generally gives the best performance. This can be
understood by the fact that such problems typically exhibit significant locality
in time and space: data that has recently been used has a high chance of being
used again in the near future; and data close (in space) to recently used data
has a high chance of being used in the near future.

\subsection{Set-associative cache}

A \emph{set-associative} cache is a nice compromise between a direct mapped
cache and a fully associative cache. In a set-associative cache, the cache is
split into chunks of $n$ cache lines. Each memory address maps deterministically
to a given chunk (according to its least significant bits) precisely as a direct
mapped cache. However, within this chunk the mapping proceeds as with a fully
associative cache with $n$ choices.

The cache eviction strategies work the same way as for a fully associative
cache.

\documentclass{beamer}

\usetheme{CambridgeUS}
\usecolortheme{orchid}

\usepackage[utf8]{inputenc}
\usepackage[T1]{fontenc}

% Paths
\newcommand{\figs}{../figs}
\newcommand{\data}{../data}
\newcommand{\code}{../code}

% URL styles
\usepackage{url}
\urlstyle{sf}

% Units
\usepackage[detect-weight=true, binary-units=true]{siunitx}
\DeclareSIUnit\flop{Flops}

% Math
\usepackage{amsmath}
\usepackage{amssymb}
\usepackage{bm}
\usepackage{nicefrac}
\newcommand{\dif}[1]{{\;\text{d}#1}}

% Graphics
\usepackage{graphicx}
\usepackage{caption}
\usepackage{subcaption}
\graphicspath{{../figs/}}

% Tikz
\usepackage{tikz}
\usetikzlibrary{positioning,shapes,arrows,calc,intersections}
\usepackage{pgfplots}
\usepgfplotslibrary{dateplot}
\pgfplotsset{compat=1.8}

% Colors
\definecolor{darkblue}{HTML}{00688B}
\definecolor{darkgreen}{HTML}{6E8B3D}
\definecolor{cadet}{HTML}{DAE1FF}
\definecolor{salmon}{HTML}{FFB08A}

% Listings
\usepackage{textcomp}
\usepackage{listings}
\lstset{
  keywordstyle=\bfseries\color{orange},
  stringstyle=\color{darkblue!80},
  commentstyle=\color{darkblue!80},
  showstringspaces=false,
  basicstyle=\ttfamily,
  upquote=true,
}
\lstdefinestyle{fortran}{
  language=Fortran,
  morekeywords={for},
  deletekeywords={status},
}
\lstdefinestyle{c}{
  language=C,
  morekeywords={include},
}
\lstdefinestyle{glsl}{
  language=C,
  morekeywords={attribute, vec2, vec3, vec4, varying, uniform, mat2, mat3, mat4},
}
\lstdefinestyle{cuda}{
  language=C,
  morekeywords={__global__, __device__, __host__},
}
\lstdefinestyle{shell}{
  language=bash,
  morekeywords={mkdir, ssh, cmake},
}

% Double hlines
\usepackage{hhline}

% Misc
\usepackage{nth}

\subtitle{TMA4280---Introduction to Supercomputing}

\begin{document}


\title{Vilje crash course}
\author{Eivind Fonn}
\institute{SINTEF ICT / NTNU}
\date{December 2015}
\maketitle

\begin{frame}
  \frametitle{Supercomputing---a users guide}
  \begin{itemize}
  \item Supercomputers use, with very few exceptions, a Unix-type operating system.
  \item Predominantly Linux these days.
  \item Students have varying degrees of Unix knowledge.
  \item This will be a crash course in using one such system (Vilje).
  \item Some basic knowledge of shell is assumed (see the crash course slides).
  \end{itemize}
\end{frame}

\begin{frame}
  \frametitle{How do I log onto the system?}
  \begin{itemize}
  \item Login is handled through \emph{secure shell} or SSH for short.
  \item On Linux and OSX computers, SSH clients are typically pre-installed.
  \item On Windows you need to install separately.
  \item One example of a client: PuTTy.
  \end{itemize}
\end{frame}

\begin{frame}[fragile]
  \frametitle{How do I log onto the system?}
  \begin{itemize}
  \item We will be using the \emph{Vilje} cluster.
  \item URL: \texttt{vilje.hpc.ntnu.no}
  \item On a Linux or OSX shell (e.g. bash):
\begin{lstlisting}[style=shell]
ssh username@vilje.hpc.ntnu.no
\end{lstlisting}
  \item This will bring you to a \emph{login terminal}. Not an actual node.
  \item Note: This machine is NOT accessible from the outside world.
  \end{itemize}
\end{frame}

\begin{frame}
  \frametitle{How do I transfer files to/from the system?}
  \begin{itemize}
  \item File transfers is performed using \emph{secure copy} or \emph{scp} for short.
  \item On a Linux or OSX computer, this comes preinstalled.
  \item On Windows you need to use special software capable of performing SSH transfers.
  \item One example of such an application is \emph{WinSCP}.
  \item Alternative approach Linux: \emph{sshfs} or \emph{sftp}.
  \item Alternative approach Windows: \emph{mindTerm} or \emph{NetDrive}.
  \item Alternative approach (both): GIT!
  \end{itemize}
\end{frame}

\begin{frame}[fragile]
  \frametitle{But I'm lazy}
  \begin{itemize}
  \item You can avoid typing your password:
\begin{lstlisting}[style=shell]
ssh-keygen
ssh-copy-id username@vilje.hpc.ntnu.no
\end{lstlisting}
  \item Putty has some kind of key mechanism but I do not know the details.
  \item You can even avoid typing the address. Put this in
    \texttt{\textasciitilde/.ssh/config}:
\begin{lstlisting}[style=shell]
Host vilje
    Hostname vilje.hpc.ntnu.no
    User eivindfo
\end{lstlisting}
    Then you can log in with
\begin{lstlisting}[style=shell]
ssh vilje
\end{lstlisting}
  \end{itemize}
\end{frame}

\begin{frame}
  \frametitle{Editing files}
  \begin{itemize}
  \item Emacs (use locally if you can, the one on Vilje is older than time)
  \item Vim (handy for using remotely, a bit of a learning curve)
  \item Nano (simpler than Vim)
  \item Gedit (nice graphical editor)
  \item Kate (same, not installed on Vilje)
  \item Notepad++ (good for Windows users, not installed on Vilje)
  \item \ldots
  \end{itemize}
\end{frame}

\begin{frame}[fragile]
  \frametitle{Graphical display (X11 forwarding)}
  If you want to run graphical programs on Vilje you have to tunnel the display
  through ssh. This is called \emph{X forwarding}.
  \begin{itemize}
  \item In Linux, it's quite easy:
\begin{lstlisting}[style=shell]
ssh -X vilje
\end{lstlisting}
  Or in your \texttt{\textasciitilde/.ssh/config}:
\begin{lstlisting}[style=shell]
ForwardX11 Yes
\end{lstlisting}
    \item In OSX, you have to start \texttt{X11.app}, then do the same.
    \item In Windows, you can use \emph{X-Win32}, which is available on progdist.
  \end{itemize}
\end{frame}

\begin{frame}[fragile]
  \frametitle{Modules}
  All software on NOTUR is offered through a \emph{modules} system. They will
  not be available to you until you load the module in question.
  \begin{itemize}
  \item List available modules:
\begin{lstlisting}[style=shell]
module avail
\end{lstlisting}
  \item Load a module:
\begin{lstlisting}[style=shell]
module load intelcomp
\end{lstlisting}
  \item Load a module with a specific version:
\begin{lstlisting}[style=shell]
module load intelcomp/13.0.1
\end{lstlisting}
  \end{itemize}
\end{frame}

\begin{frame}[fragile]
  \frametitle{Modules}
  Some relevant modules for this course:
  \begin{itemize}
    \item \texttt{intelcomp}: Intel's compilers (\texttt{icc}, \texttt{icpc} and
      \texttt{ifort}).
    \item \texttt{mpt}: SGI's implementation of MPI (\emph{Message Passing
        Toolkit}).
    \item \texttt{cmake}
    \item \texttt{mercurial}: If you like DVCS but not git.
  \end{itemize}

  Note that if you use CMake to build your programs, you have to tell it to use
  Intel's compilers:
\begin{lstlisting}[style=shell]
mkdir build
cd build
CXX=icpc CC=icc FC=ifort cmake ..
\end{lstlisting}
\end{frame}

\begin{frame}[fragile]
  \frametitle{Running jobs on Vilje}
  After compiling your program, you have to write a \emph{job script}. Example
  (the \emph{pi} program):
\begin{lstlisting}[style=shell]
  #!/bin/bash

  #PBS -N pi
  #PBS -A ntnu603
  #PBS -l walltime=00:01:00
  #PBS -l select=2:ncpus=32:mpiprocs=16

  cd $PBS_O_WORKDIR
  module load mpt
  mpiexec_mpt ./pi 1000000
\end{lstlisting}
%$
\end{frame}

\begin{frame}[fragile]
  \frametitle{Running jobs on Vilje}
\begin{lstlisting}[style=shell]
  #PBS -N pi
\end{lstlisting}
  My job is called ``pi''.
\begin{lstlisting}[style=shell]
  #PBS -A ntnu603
\end{lstlisting}
  The time spent executing this job should be charged to ntnu603.
\begin{lstlisting}[style=shell]
  #PBS -l walltime=00:01:00
\end{lstlisting}
  The walltime limit for this job is one minute.
\begin{lstlisting}[style=shell]
  #PBS -l select=2:ncpus=32:mpiprocs=16
\end{lstlisting}
  I want two units of 32 CPUs each (two nodes, that is) and I want 16 processes
  on each of them (32 in total). On Vilje, \texttt{ncpus} should \emph{always}
  be equal to 32.
\end{frame}

\begin{frame}[fragile]
  \frametitle{Running jobs on Vilje}
\begin{lstlisting}[style=shell]
  cd $PBS_O_WORKDIR
\end{lstlisting}
%$
  Ensure that we are in the correct directory. This should always be in your job
  script.
\begin{lstlisting}[style=shell]
  module load mpt
\end{lstlisting}
  Make sure the \texttt{mpt} module is loaded so that the \texttt{mpiexec\_mpt}
  command is available.
\begin{lstlisting}[style=shell]
  mpiexec_mpt ./pi 1000000
\end{lstlisting}
  Run the program.
\end{frame}

\begin{frame}[fragile]
  \frametitle{Running jobs on Vilje}
  Submit a job using \texttt{qsub}:
\begin{lstlisting}[style=shell]
  qsub job.sh
  5723717.service2
\end{lstlisting}
  \texttt{qsub} will reply with a job ID number. You can ask for the status of
  your job with
\begin{lstlisting}[style=shell]
  qstat -f 5723717.service2
\end{lstlisting}
  or see a list of all jobs running and queued
\begin{lstlisting}[style=shell]
  qstat
\end{lstlisting}
\end{frame}

\begin{frame}[fragile]
  \frametitle{Running jobs on Vilje}
  When the program has completed, the accumulated output will be written to
  files in the same folder you launched it from.
\begin{lstlisting}[style=shell]
  ls
  job.sh  pi  pi.c  pi.e5723717  pi.o5723717
\end{lstlisting}
  The \texttt{e}-file contains stderr (empty?) and the \texttt{o}-file contains
  output from stdout (the most interesting one).
\begin{lstlisting}[style=shell,basicstyle=\ttfamily\footnotesize]
  cat pi.o5723717
  Agent pid 21651
  pi=3.141593e+00, error=8.437695e-14, duration=2.177000e-03
  Start Epilogue v3.0.1 Wed Jan 27 14:18:27 CET 2016
  clean up
  End Epilogue v3.0.1 Wed Jan 27 14:18:28 CET 2016
\end{lstlisting}
\end{frame}

\begin{frame}[fragile]
  \frametitle{Other PBS options}
\begin{lstlisting}[style=shell]
  #PBS -o stdout
  #PBS -e stderr
\end{lstlisting}
I want my output files to have more sensible names.
\begin{lstlisting}[style=shell]
  #PBS -m abe
\end{lstlisting}
I want an e-mail notification when the job starts (\texttt{b}), ends
(\texttt{e}) or if it aborts (\texttt{a}).
\begin{lstlisting}[style=shell]
  #PBS -M some@where.com
\end{lstlisting}
\ldots and this is where that e-mail should be sent to.
\begin{lstlisting}[style=shell]
  #PBS -l ...:ompthreads=16
\end{lstlisting}
I want 16 OpenMP threads per process.
\end{frame}

\begin{frame}
  \frametitle{More information}
  \begin{center}
    The NTNU HPC Wiki has a very good user guide. \\~\\
    https://www.hpc.ntnu.no/display/hpc/User+Guide
  \end{center}
\end{frame}

\end{document}


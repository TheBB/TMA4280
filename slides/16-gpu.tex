\documentclass{beamer}

\usetheme{CambridgeUS}
\usecolortheme{orchid}

\usepackage[utf8]{inputenc}
\usepackage[T1]{fontenc}

% Math
\usepackage{amsmath}
\usepackage{amssymb}
\usepackage{bm}

% Graphics
\usepackage{graphicx}
\usepackage{caption}
\usepackage{subcaption}
\graphicspath{{../figs/}}

% Colors
\definecolor{darkblue}{HTML}{00688B}
\definecolor{darkgreen}{HTML}{6E8B3D}
\definecolor{cadet}{HTML}{DAE1FF}
\definecolor{salmon}{HTML}{FFB08A}

% Listings
\usepackage{textcomp}
\usepackage{listings}
\lstset{
  keywordstyle=\bfseries\color{orange},
  stringstyle=\color{darkblue!80},
  commentstyle=\color{darkblue!80},
  showstringspaces=false,
  basicstyle=\ttfamily,
  upquote=true,
}
\lstdefinestyle{fortran}{
  language=Fortran,
  morekeywords={for},
  deletekeywords={status},
}
\lstdefinestyle{c}{
  language=C,
  morekeywords={include},
}
\lstdefinestyle{shell}{
  language=bash,
}

\subtitle{TMA4280---Introduction to Supercomputing}

\begin{document}


\title{Graphics programming}
\author{Eivind Fonn}
\institute{SINTEF ICT / NTNU}
\date{March 2016}
\maketitle

\begin{frame}
  \frametitle{Graphics programming}
  \begin{itemize}
  \item Graphics programming is a field unto itself. We won't study all of it.
  \item A 3D scene involves a large number of polygons (usually triangles), with
    associated colors.
  \item To render it, all points must be transformed from \emph{model space} to
    \emph{image space}, and the correct color value must be computed for each
    pixel.
  \item Several such images must be ready per second.
  \item Since there are many polygons and pixels, we need a processor with
    extreme parallel capabilities.
  \end{itemize}
\end{frame}

\begin{frame}
  \frametitle{Vertex transformation}
  \begin{itemize}
  \item Vertices are transformed from model space to image space by multiplying
    with an \emph{MVP} matrix.
    \[
      \bm v_p = \bm P \bm V \bm M \bm v_m
    \]
  \item $\bm M$: The \emph{model} matrix transforms from the local coordinate
    system of a model to the global coordinate system of the scene.
  \item $\bm V$: The \emph{view} matrix transforms from the global coordinate
    system to a coordinate system with the camera at the origin.
  \item $\bm P$: The \emph{perspective} matrix transforms from the camera system
    to image coordinates. It depends on camera parameters like field of view.
  \end{itemize}
\end{frame}

\begin{frame}
  \frametitle{Vertex transformation}
  \begin{itemize}
  \item Note that each vertex is a four-dimensional vector, and each matrix is
    $4 \times 4$.
  \item This is necessary to allow us to do affine transformations, such as
    translations:
    \[
      \begin{pmatrix}
        1 &   &   & t_x \\
        & 1 &   & t_y \\
        &   & 1 & t_z \\
        &   &   & 1
      \end{pmatrix}
      \begin{pmatrix} v_x \\ v_y \\ v_z \\ 1 \end{pmatrix}
      =
      \begin{pmatrix} v_x + t_x \\ v_y + t_y \\ v_z + t_z \\ 1 \end{pmatrix}
    \]
  \item All necessary transformations can then be encoded as matrix operations.
  \end{itemize}
\end{frame}

\begin{frame}
  \frametitle{OpenGL}
  \begin{itemize}
  \item OpenGL (Open Graphics Library) is an API for graphics programming.
  \item It's a \emph{standard} which must be \emph{implemented} by the driver of
    a GPU. Therefore there are many OpenGL implementations.
  \item The standard is developed by the Khronos Group consortium, with members
    such as AMD, Apple, Google, Intel, Nvidia, Samsung, etc.
  \item Unlike Direct3D, OpenGL is cross platform.
  \item Due to be replaced by a newer standard called \emph{Vulkan} which was
    released very recently.
  \end{itemize}
\end{frame}

\begin{frame}[fragile]
  \frametitle{``Old'' style OpenGL}
  \begin{lstlisting}[style=c]
  glFrustum(left, right, bottom, top, near, far);
  glRotatef(angle, nx, ny, nz);
  glTranslatef(dx, dy, dz);
  // Other matrix operations...

  glBegin(GL_TRANGLES);
  for (...) {
    glColor3f(r, g, b);
    glVertex3f(x, y, z);
  }
  glEnd();
  \end{lstlisting}
\end{frame}

\begin{frame}
  \frametitle{``Old'' style OpenGL}
  \begin{itemize}
  \item The code runs, as usual, on the CPU. Each call to \texttt{glSomething}
    makes something happen on the GPU.
  \item Since each vertex requires at least one function call on the CPU
    \texttt{glVertex}, this style is highly bound by CPU performance.
  \item Worse, OpenGL is highly stateful, so there's no good way to parallelize
    on the CPU side either.
  \item This style is usually called the \emph{fixed function pipeline}.
  \end{itemize}
\end{frame}

\begin{frame}
  \frametitle{Modern OpenGL}
  \begin{itemize}
  \item Modern OpenGL instead uses the \emph{programmable pipeline}.
  \item This involves writing programs that run on the GPU, which operate on
    data in particular ways.
  \item These programs are called \emph{shaders} and are written in the
    \emph{GLSL} language (OpenGL Shading Language). It looks a lot like C and is
    compiled at runtime by the driver.
  \item There are many kinds of shaders. In particular:
    \begin{itemize}
    \item Vertex shaders compute the MVP transformation.
    \item Fragment shaders compute the color of a pixel.
    \end{itemize}
  \item The data are allocated and stored on the GPU itself, in its own memory.
  \end{itemize}
\end{frame}

\begin{frame}[fragile]
  \frametitle{Vertex shader}
  This is a simple vertex shader.
  \begin{lstlisting}[style=glsl]
  attribute vec3 in_coord;
  attribute vec3 in_color;
  uniform mat4 transform;
  varying vec3 f_color;

  void main(void) {
    gl_Position = transform * vec4(coord, 1.0);
    f_color = in_color;
  }
  \end{lstlisting}
\end{frame}

\begin{frame}
  \frametitle{Vertex shader}
  \begin{itemize}
  \item The shader declares two \emph{attributes} called \texttt{in\_coord} and
    \texttt{in\_color}. They are 3D vectors and represent the position and
    color of a vertex.
  \item Additionally the actual transformation matrix is declared as a
    \emph{uniform}, which means it is constant.
  \item The \emph{varying} declaration of \texttt{f\_color} means that it is a
    parameter to the fragment shader, which will be interpolated between
    vertices.
  \item Finally, the vertex shader defines a function \texttt{main} whose
    purpose it is to set the varying \texttt{f\_color} and the four-dimensional
    vector \texttt{gl\_Position} (the final position of the vertex).
  \end{itemize}
\end{frame}

\begin{frame}[fragile]
  \frametitle{Fragment shader}
  The fragment shader is simpler.
  \begin{lstlisting}[style=glsl]
  varying vec3 f_color;

  void main(void) {
    gl_FragColor = vec3(f_color, 1.0);
  }
  \end{lstlisting}
\end{frame}

\begin{frame}[fragile]
  \frametitle{A GLSL program}
  \begin{lstlisting}[style=c, basicstyle=\ttfamily\footnotesize]
  GLuint vs = glCreateShader(GL_VERTEX_SHADER);
  glShaderSource(vs, 1, vs_source, NULL);
  glCompileShader(vs);

  GLuint fs = glCreateShader(GL_FRAGMENT_SHADER);
  glShaderSource(fs, 1, fs_source, NULL);
  glCompileShader(fs);

  GLuint program = glCreateProgram();
  glAttachShader(program, vs);
  glAttachShader(program, fs);
  glLinkProgram(program);
  \end{lstlisting}
\end{frame}

\begin{frame}[fragile]
  \frametitle{Storing data in graphics memory}
  \begin{lstlisting}[style=c, basicstyle=\ttfamily\footnotesize]
  GLuint coords;
  glGenBuffers(1, &coords);
  glBindBuffer(GL_ARRAY_BUFFER, coords);
  glBufferData(GL_ARRAY_BUFFER, size, ptr, GL_STATIC_DRAW);

  GLuint colors;
  glGenBuffers(1, &colors);
  glBindBuffer(GL_ARRAY_BUFFER, colors);
  glBufferData(GL_ARRAY_BUFFER, size, ptr, GL_STATIC_DRAW);
  \end{lstlisting}
\end{frame}

\begin{frame}[fragile]
  \frametitle{Rendering a frame}
  \begin{lstlisting}[style=c, basicstyle=\ttfamily\scriptsize]
glUseProgram(program);
GLint a_coord = glGetAttribLocation(program, "in_coord");
GLint a_color = glGetAttribLocation(program, "in_color");
GLint u_trans = glGetUniformLocation(program, "transform");

glEnableVertexAttribArray(a_coord);
glBindBuffer(GL_ARRAY_BUFFER, coords);
glVertexAttribPointer(a_coord, 3, GL_FLOAT, GL_FALSE, 0, 0);

glEnableVertexAttribArray(a_color);
glBindBuffer(GL_ARRAY_BUFFER, colors);
glVertexAttribPointer(a_color, 3, GL_FLOAT, GL_FALSE, 0, 0);

glUniformMatrix4fv(u_trans, 1, GL_FALSE, ptr);

glDrawArrays(GL_TRIANGLES, 0, 3);

glDisableVertexAttribArray(a_color);
glDisableVertexAttribArray(a_coord);
  \end{lstlisting}
\end{frame}

\begin{frame}
  \frametitle{Rendering a frame}
  \begin{itemize}
  \item For each frame, the CPU must
    \begin{itemize}
    \item Enable each attribute in the GLSL program
    \item Bind each buffer with source data
    \item Point each attribute to the correct buffer
    \item Calculate and set the value of the uniform transformation matrix
    \item Call the drawing function
    \item Disable the attributes
    \end{itemize}
  \item It seems like a lot but it scales very well with the size of the models.
  \item All the other things can happen once, during setup:
    \begin{itemize}
    \item Compiling and linking the GLSL program
    \item Allocating GPU memory and copying data
    \end{itemize}
  \end{itemize}
\end{frame}

\begin{frame}
  \frametitle{GPU design}
  \begin{itemize}
  \item The needs of graphics programming dictate how a GPU is designed.
  \item Extreme parallelism: a GPU can have thousands of cores.
  \item GPUs are optimized for performing the same operations on each core, in a
    SIMD-like fashion.
  \item Each core is quite minimal: only an integer and a floating point unit,
    for example.
  \end{itemize}
\end{frame}

\begin{frame}
  \frametitle{Modern OpenGL}
  \begin{itemize}
  \item The programmable pipeline is quite flexible.
  \item Drawing pictures is a noble goal but we want more.
  \item Compute shaders are designed for arbitrary computations on the GPU.
  \item They can be attached to a GLSL program, but take no input and produce no
    output. A compute shader must fetch the data it needs on its own.
  \item Compute shaders are not invoked during the regular rendering pipeline.
    They must be explicitly invoked using \texttt{glDispatchCompute} from the
    CPU.
  \item All this makes compute shaders somewhat awkward. (As if OpenGL wasn't
    already awkward enough.)
  \end{itemize}
\end{frame}

\begin{frame}
  \frametitle{CUDA and OpenCL}
  \begin{itemize}
  \item Enter: CUDA and OpenCL.
  \item These are APIs for running more general-purpose programs on GPUs.
  \item OpenCL is a Khronos standard while CUDA is Nvidia's own model.
  \item AMD GPU drivers typically support OpenCL 2.0 while Nvidia drivers
    support OpenCL 1.1 (possibly to push people to CUDA).
  \item Here we will be looking at CUDA, mostly because I use Nvidia GPUs.
    (Nvidia drivers on Linux have been better than AMD drivers for a long time.)
  \end{itemize}
\end{frame}

\begin{frame}[fragile]
  \frametitle{CUDA}
  \begin{itemize}
  \item A CUDA program contains that executes on the CPU (the \emph{host}) and
    on the GPU (the \emph{device}).
  \item The CUDA compiler (\texttt{nvcc}) functions as a wrapper that compiles
    the device code on its own and passes the host code to a regular C compiler.
    It is invoked like any other compiler.
  \item The resulting program can be run like any other program.
  \end{itemize}
\end{frame}

\begin{frame}[fragile]
  \frametitle{Hello CUDA}
  \begin{lstlisting}[style=cuda]
  __global__ void hello() { }

  int main() {
    hello<<<1,1>>>();
    return 0;
  }
  \end{lstlisting}
  \begin{itemize}
  \item The \texttt{\_\_global\_\_} qualifier signals that the \texttt{hello}
    function runs on the device, but can be called from the host.
  \item A function with \texttt{\_\_device\_\_} runs on, and is called from the
    device.
  \item To call a device function from host code, we invoke it in a particular
    way, with triple angle brackets. We'll come back to this later.
  \end{itemize}
\end{frame}

\begin{frame}[fragile]
  \frametitle{Addition on the GPU}
  \begin{lstlisting}[style=cuda, basicstyle=\ttfamily\scriptsize]
  __global__ void add(int *a, int *b, int *c) {
    *c = *a + *b;
  }

  int main() {
    int *d_a, *d_b, *d_c;
    cudaMalloc((void**)&d_a, sizeof(int));
    // Same for b and c

    int a = 2, b = 7, c;
    cudaMemcpy(d_a, &a, sizeof(int), cudaMemcpyHostToDevice);
    // Same for b

    add<<<1,1>>>(d_a, d_b, d_c);

    cudaMemcpy(d_c, &c, sizeof(int), cudaMemcpyDeviceToHost);
    cudaFree(d_a); cudaFree(d_b); cudaFree(d_c);

    printf("%d\n", c);
    return 0;
  }
  \end{lstlisting}
\end{frame}

\begin{frame}
  \frametitle{Memory}
  \begin{itemize}
  \item Copying data to the GPU involves allocating memory there (with
    \texttt{cudaMalloc}) and copying from the host (with \texttt{cudaMemcpy}).
  \item These functions are vaguely similar to regular C \texttt{malloc} and
    \texttt{memcpy}.
  \item Note that we have pointers to device memory stored in host variables
    (\texttt{d\_a} etc.) This is fine, so long as you don't dereference device
    pointers in host code.
  \item \texttt{\_\_global\_\_} functions must have \texttt{void} return types,
    so we must copy the result back.
  \end{itemize}
\end{frame}

\begin{frame}
  \frametitle{Parallelism terms}
  \begin{itemize}
  \item A \texttt{\_\_global\_\_} function is called a \emph{kernel}.
  \item A kernel, when invoked, has a certain number of \emph{blocks} organized
    in a \emph{grid}. The grid may be as much as three-dimensional, but here we
    will stay with one-dimensional grids.
  \item Each block, in turn, contain a certain number of \emph{threads} which
    are also organized in a grid, which may be of different size than the
    previously mentioned grid.
  \item Threads are grouped together in \emph{warps} for execution on the GPU.
    The size of a warp depends on the microarchitecture of the GPU in question.
    E.g. for Nvidia Fermi GPUs, a warp has 32 threads.
  \end{itemize}
\end{frame}

\begin{frame}
  \frametitle{Parallelism}
  \begin{itemize}
  \item The two numbers in angle brackets specify the number of blocks in the
    grid, and the number of threads per block.
  \item So far, we have seen code where a single block with a single thread is
    executed.
  \item Let's try some actual parallelism.
  \end{itemize}
\end{frame}

\begin{frame}[fragile]
  \frametitle{Vector addition on the GPU (blocks)}
  \begin{lstlisting}[style=cuda, basicstyle=\ttfamily\scriptsize]
  __global__ void add(int *a, int *b, int *c) {
    c[blockIdx.x] = a[blockIdx.x] + b[blockIdx.x];
  }


  int main() {
    int *a, *b, *c, N;  // Assume set
    int size = N * sizeof(int);

    int *d_a, *d_b, *d_c;
    cudaMalloc((void**)&d_a, size);  // Same for b and c

    cudaMemcpy(d_a, &a, size, cudaMemcpyHostToDevice);
    cudaMemcpy(d_b, &b, size, cudaMemcpyHostToDevice);

    add<<<N,1>>>(d_a, d_b, d_c);

    cudaMemcpy(d_c, &c, size, cudaMemcpyDeviceToHost);
    cudaFree(d_a); cudaFree(d_b); cudaFree(d_c);

    return 0;
  }
  \end{lstlisting}
\end{frame}

\begin{frame}[fragile]
  \frametitle{Vector addition on the GPU (threads)}
  \begin{lstlisting}[style=cuda, basicstyle=\ttfamily\scriptsize]
  __global__ void add(int *a, int *b, int *c) {
    c[threadIdx.x] = a[threadIdx.x] + b[threadIdx.x];
  }


  int main() {
    int *a, *b, *c, N;  // Assume set
    int size = N * sizeof(int);

    int *d_a, *d_b, *d_c;
    cudaMalloc((void**)&d_a, size);  // Same for b and c

    cudaMemcpy(d_a, &a, size, cudaMemcpyHostToDevice);
    cudaMemcpy(d_b, &b, size, cudaMemcpyHostToDevice);

    add<<<1,N>>>(d_a, d_b, d_c);

    cudaMemcpy(d_c, &c, size, cudaMemcpyDeviceToHost);
    cudaFree(d_a); cudaFree(d_b); cudaFree(d_c);

    return 0;
  }
  \end{lstlisting}
\end{frame}

\begin{frame}[fragile]
  \frametitle{Vector addition on the GPU (both)}
  \begin{lstlisting}[style=cuda, basicstyle=\ttfamily\scriptsize]
  __global__ void add(int *a, int *b, int *c) {
    int idx = threadIdx.x + blockIdx.x * blockDim.x;
    c[idx] = a[idx] + b[idx];
  }

  int main() {
    int *a, *b, *c, M, N;  // Assume set
    int size = M * N * sizeof(int);

    int *d_a, *d_b, *d_c;
    cudaMalloc((void**)&d_a, size);  // Same for b and c

    cudaMemcpy(d_a, &a, size, cudaMemcpyHostToDevice);
    cudaMemcpy(d_b, &b, size, cudaMemcpyHostToDevice);

    add<<<M,N>>>(d_a, d_b, d_c);

    cudaMemcpy(d_c, &c, size, cudaMemcpyDeviceToHost);
    cudaFree(d_a); cudaFree(d_b); cudaFree(d_c);

    return 0;
  }
  \end{lstlisting}
\end{frame}

\begin{frame}
  \frametitle{Blocks, threads and warps}
  \begin{itemize}
  \item What are the purposes of blocks and threads?
  \item Unlike blocks, threads can communicate and synchronize between each
    other. They can also share memory.
  \item In a way, a block is like a process in MPI and a thread is like a thread
    in OpenMP.
  \item Like in OpenMP, there is a limit to the number of threads per block. On
    current GPUs, that limit is 1024. Practical concerns may make this limit
    lower, depending on your problem.
  \end{itemize}
\end{frame}

\begin{frame}
  \frametitle{Blocks, threads and warps}
  \begin{itemize}
  \item A \emph{warp} is a group of 32 threads executed together on a single
    multiprocessor.
  \item Each thread in a warp always execute the same instruction. If different
    threads take different execution paths, all threads in the warp execute
    \emph{all} paths, only with some of them disabled when appropriate.
  \item A typical performance problem is if different threads in a warp take
    different execution paths.
  \item Try to write your code so that threads close to each other do not take
    diverging execution paths.
  \end{itemize}
\end{frame}

\input{postamble}

\documentclass{beamer}

\usetheme{CambridgeUS}
\usecolortheme{orchid}

\usepackage[utf8]{inputenc}
\usepackage[T1]{fontenc}

% Math
\usepackage{amsmath}
\usepackage{amssymb}
\usepackage{bm}

% Graphics
\usepackage{graphicx}
\usepackage{caption}
\usepackage{subcaption}
\graphicspath{{../figs/}}

% Colors
\definecolor{darkblue}{HTML}{00688B}
\definecolor{darkgreen}{HTML}{6E8B3D}
\definecolor{cadet}{HTML}{DAE1FF}
\definecolor{salmon}{HTML}{FFB08A}

% Listings
\usepackage{textcomp}
\usepackage{listings}
\lstset{
  keywordstyle=\bfseries\color{orange},
  stringstyle=\color{darkblue!80},
  commentstyle=\color{darkblue!80},
  showstringspaces=false,
  basicstyle=\ttfamily,
  upquote=true,
}
\lstdefinestyle{fortran}{
  language=Fortran,
  morekeywords={for},
  deletekeywords={status},
}
\lstdefinestyle{c}{
  language=C,
  morekeywords={include},
}
\lstdefinestyle{shell}{
  language=bash,
}

\subtitle{TMA4280---Introduction to Supercomputing}

\begin{document}


\title{PETSc}
\author{Eivind Fonn}
\institute{SINTEF ICT / NTNU}
\date{December 2015}
\maketitle

\begin{frame}
  \frametitle{Linear algebra packages}
  \begin{itemize}
  \item So far: BLAS and LAPACK (dense problems, forward and backward
    operations).
  \item There are also packages for sparse linear algebra.
    \begin{itemize}
    \item MUMPS: MUltifrontal Massively Parallel sparse direct Solver, not to be
      confused with the Massachusetts general hospital Utility Multi-Programming
      System, which makes COBOL look like fun, nor to be confused with the
      disease (which also makes COBOL look like fun).
    \item UMFPACK: Unsymmetric MultiFrontal method
    \item SuperLU: Sparse, direct solver (comes in threaded and distributed
      variants but no hybrid)
    \item HYPRE: Sparse, iterative solvers and preconditioners
    \end{itemize}
  \end{itemize}
\end{frame}

\begin{frame}
  \frametitle{Linear algebra packages}
  Some linear algebra packages have more than just solvers, but contain other
  tools that are helpful in scientific computing as well.
  \begin{itemize}
  \item Trilinos: Solvers, finite element libraries and more
  \item PETSc: The Portable Extensible Toolkit for Scientific computing.
  \end{itemize}
  It's possible to use Trilinos trough PETSc and probably also the other way
  around. In the following we will focus on PETSc.
\end{frame}

\begin{frame}
  \frametitle{PETSc}
  \begin{itemize}
  \item PETSc (Pet-see) is an open-source scientific computing library written
    in C.
  \item In spite of being written in C, it calls itself object oriented.
  \item There are interfaces available for Fortran, Python and Matlab.
  \item PETSc has great support for distributed programming. Also available is a
    form of threaded support and (upcoming) hybrid.
  \item It can make use of Nvidia GPUs through CUDA.
  \item Direct and iterative linear solvers for sparse matrices.
  \end{itemize}
\end{frame}

\begin{frame}[fragile]
  \frametitle{Installing PETSc}
  \begin{itemize}
  \item PETSc is packaged for Ubuntu:
    \begin{lstlisting}[style=shell]
sudo apt-get install libpetsc-dev
    \end{lstlisting}
  \item On Vilje it is available as a module:
    \begin{lstlisting}[style=shell]
module load petsc
    \end{lstlisting}
  \item Both these packages are available with MPI support.
  \item On Vilje (and on Ubuntu 14.04) the default PETSc version is 3.4. The
    latest release is 3.6.
  \item At Sintef, we compile PETSc manually.
  \end{itemize}
\end{frame}

\begin{frame}[fragile]
  \frametitle{Building PETSc}
  \begin{itemize}
  \item Once PETSc has been downloaded, you should define two environment
    variables.
    \begin{itemize}
    \item \texttt{PETSC\_DIR}: The path to the main library headers.
    \item \texttt{PETSC\_ARCH}: The name of your build configuration. You can
      make something up.
    \end{itemize}
  \item These are defined for you by the module on Vilje, but on e.g. Ubuntu you
    have to set them manually (last time I checked).
  \end{itemize}
\end{frame}

\begin{frame}[fragile]
  \frametitle{Building PETSc}
  Building PETSc can be tricky. It does not use CMake. Here's an example
  configuration command.
  \begin{lstlisting}[basicstyle=\ttfamily\scriptsize]
  PETSC_DIR=/home/akva/kode/petsc-3.4.2
  PETSC_ARCH=linux-gnu-cxx-opt
  ./configure --with-precision=double
              --with-scalar-type=real
              --with-debugging=0
              --COPTFLAGS=-O3 --CXXOPTFLAGS=-O3
              --FCOPTFLAGS=-O3
              --with-blas-lib=/usr/lib/libblas.a
              --with-lapack-lib=/usr/lib/liblapack.a
              --with-64-bit-indices=0 --with-clanguage=c++
              --with-mpi=1 --LIBS=-ldl
              --with-hypre=1 --with-umfpack=1
              --with-ml=1
              --with-superlu=1
  \end{lstlisting}
\end{frame}

\begin{frame}
  \frametitle{Linking to PETSc}
  \begin{itemize}
  \item As PETSc is a big package, linking against it in your program is not just
    a matter of adding \texttt{-lpetsc}.
  \item Imperative to use helpers such as CMake for this.
  \item PETSc changed to a CMake based build system in version 3.2. This makes
    our job a lot easier. (You should still use the basic configuration command,
    however.)
  \item By default, there is no support for PETSc in CMake 2.8. However,
    CMake is easy to extend.
  \end{itemize}
\end{frame}

\begin{frame}[fragile]
  \frametitle{Finding PETSc}
  \begin{itemize}
  \item We have seen how CMake finds a library
    \begin{lstlisting}
      find_package(Foo REQUIRED)
    \end{lstlisting}
  \item What happens is that CMake runs a file called \texttt{FindFoo.cmake}.
    This is just a file with regular CMake commands, which in the end sets
    certain variables, which we can use afterwards.
  \item We can write files like this ourselves.
  \end{itemize}
\end{frame}

\begin{frame}[fragile]
  \frametitle{Finding PETSc with CMake}
  \begin{lstlisting}[basicstyle=\ttfamily\scriptsize]
find_path(
  PETSC_INCLUDE_DIR
  NAMES petsc.h
  PATHS /usr/include/petsc/
        $ENV{PETSC_DIR}/include)
find_path(
  PETSCCONF_INCLUDE_DIR
  NAMES petscconf.h
  PATHS /usr/include/petsc/
        $ENV{PETSC_DIR}/$ENV{PETSC_ARCH}/include)

include($ENV{PETSC_DIR}/$ENV{PETSC_ARCH}/conf/PETScConfig.cmake)
find_library(PETSC_LIB_PETSC petsc
             HINTS $ENV{PETSC_DIR}/$ENV{PETSC_ARCH}/lib)
set(PETSC_LIBRARIES ${PETSC_LIB_PETSC} ${PETSC_PACKAGE_LIBS})
set(PETSC_INCLUDES ${PETSC_INCLUDE_DIR} ${PETSCCONF_INCLUDE_DIR}
                   ${PETSC_PACKAGE_INCLUDES})
  \end{lstlisting}
\end{frame}

\begin{frame}[fragile]
  \frametitle{Finding PETSc with CMake}
  \begin{itemize}
  \item Before we can use this rule, we have to tell CMake where to find
    \texttt{FindPETSc.cmake}. (How to find the finder.)
  \item Typically such files are stored in \texttt{cmake/Modules} in your
    project, so this is a typical line to find in CMake files:
    \begin{lstlisting}
set(CMAKE_MODULE_PATH
    ${CMAKE_MODULE_PATH}
    ${CMAKE_SOURCE_DIR}/cmake/Modules)
    \end{lstlisting}
  \end{itemize}
\end{frame}

\begin{frame}[fragile]
  \frametitle{Finished CMakeLists.txt}
  \begin{lstlisting}
set(CMAKE_MODULE_PATH
    ${CMAKE_MODULE_PATH}
    ${CMAKE_SOURCE_DIR}/cmake/Modules)

find_package(PETSc REQUIRED)

include_directories(${PETSC_INCLUDES})

add_executable(petsc-test main.c)
target_link_libraries(petsc-test
                      ${PETSC_LIBRARIES})
  \end{lstlisting}
\end{frame}

\begin{frame}
  \frametitle{PETSc conventions}
  \begin{itemize}
  \item PETSc is a large library, so it needs conventions to keep it organized.
  \item Methods and symbols have prefixes which relate to their category.
    \begin{itemize}
    \item Vec: Vectors
    \item Mat: Matrices
    \item KSP: Linear solves (Krylov space)
    \item PC: Preconditioners
    \item Etc\ldots
    \end{itemize}
  \end{itemize}
\end{frame}

\begin{frame}
  \frametitle{PETSc Vectors}
  \begin{itemize}
  \item A vector is stored in an opaque data structure called a \texttt{Vec}.
  \item Vectors can have different types, such as
    \begin{itemize}
    \item seq: Sequential (normal)
    \item mpi: Distributed
    \item pthread: Threaded
    \item cusp: CUDA (Nividia GPU) format
    \end{itemize}
  \item The type and its implementation is hidden from the user: it's an
    implementation detail.
  \item There is \emph{no} direct data access. Everything goes through
    functions. This helps to abstract away implementation details for different
    types.
  \end{itemize}
\end{frame}

\begin{frame}
  \frametitle{PETSc Matrices}
  \begin{itemize}
  \item A matrix is stored in an opaque data structure called a \texttt{Mat}.
  \item Matrices can also have different types,
    \begin{itemize}
    \item seqdense: Sequential (normal) dense
    \item seqaij: Sequential (normal) sparse
    \item mpiaij: Distributed sparse
    \item aijcusp: CUDA (Nvidia GPU) sparse
    \end{itemize}
  \item We will play around with seqaij and mpiaij matrices.
  \end{itemize}
\end{frame}

\begin{frame}
  \frametitle{Poisson solver in PETSc}
  \begin{itemize}
  \item As is common with C libraries like this, much of our code will be
    initialization.
  \item PETSc has tools for finite difference methods, but we will avoid them.
  \item We focus here only on the setup of the vector, the matrix and the
    solution of the linear system.
  \end{itemize}
\end{frame}

\begin{frame}[fragile]
  \frametitle{Vector setup}
  \begin{lstlisting}[style=c]
  Vec b;

  // Use PETSC_COMM_SELF to get a seq vector
  VecCreate(PETSC_COMM_WORLD, &b);

  // Local and global sizes
  VecSetSizes(b, PETSC_DECIDE, m*m);

  double hh = 1.0 / n / n;
  for (size_t j = 0; j < m*m; j++)
    VecSetValue(b, j, hh, INSERT_VALUES);
  \end{lstlisting}
\end{frame}

\begin{frame}[fragile]
  \frametitle{Matrix setup}
  \begin{lstlisting}[style=c, basicstyle=\ttfamily\scriptsize]
  Mat A;
  MatCreate(PETSC_COMM_WORLD, &A);
  MatSetType(A, MATSEQAIJ);
  MatSetSizes(A, PETSC_DECIDE, PETSC_DECIDE, m*m, m*m);

  // Diagonal
  for (size_t i = 0; i < m*m, i++)
    MatSetValue(A, i, i, 4.0, INSERT_VALUES);

  // L-R coupling
  for (size_t i = 0; i < m*m - 1, i++) {
    if (i % m != m-1)
      MatSetValue(A, i, i+1, -1.0, INSERT_VALUES);
    if (i % m)
      MatSetValue(A, i, i-1, -1.0, INSERT_VALUES);
  }

  // U-D coupling
  for (size_t i = m; i < m*m, i++) {
    MatSetValue(A, i, i-m, -1.0, INSERT_VALUES);
    MatSetValue(A, i-m, i, -1.0, INSERT_VALUES);
  }
  \end{lstlisting}
\end{frame}

\begin{frame}[fragile]
  \frametitle{Synchronization}
  The vector and matrix must be \emph{assembled} before we can use them. This
  might involve communication.
  \begin{lstlisting}[style=c]
    VecAssemblyBegin(b);
    VecAssemblyEnd(b);
    MatAssemblyBegin(A, MAT_FINAL_ASSEMBLY);
    MatAssemblyEnd(A, MAT_FINAL_ASSEMBLY);
  \end{lstlisting}
\end{frame}

\begin{frame}[fragile]
  \frametitle{Solver setup}
  \begin{lstlisting}[style=c]
  KSP sol;
  KSPCreate(PETSC_COMM_WORLD, &sol);

  KSPSetType(ksp, "cg");
  KSPSetTolerances(ksp, 1e-10, 1e-10, 1e6, 10000);
  KSPSetOperators(ksp, A, A);

  PC pc;
  KSPGetPC(ksp, &pc);
  PCSetType(pc, "ilu");
  PCSetFromOptions(pc);
  PCSetUp(pc);

  KSPSetFromOptions(ksp);
  KSPSetUp(ksp);
  \end{lstlisting}
\end{frame}

\begin{frame}[fragile]
  \frametitle{Solving}
  \begin{itemize}
  \item To solve,
    \begin{lstlisting}[style=c]
  Vec x;
  VecDuplicate(b, &x);
  KSPSolve(ksp, b, x);
    \end{lstlisting}
  \item You will probably find that the solver performs rather poorly, both in
    serial and parallel.
  \item Reasons for this include
    \begin{enumerate}
    \item We fill the vector element by element,
    \item We fill the matrix element by element,
    \item We haven't declared the sparsity pattern of the matrix, and
    \item All processes fill all elements.
    \end{enumerate}
  \end{itemize}
\end{frame}

\begin{frame}[fragile]
  \frametitle{Fill the whole vector at once}
  \begin{lstlisting}[style=c]
  PetscInt low, high;
  VecGetOwnershipRange(b, &low, &high);

  PetscInt indices[high-low];
  double vals[high-low];
  for (size_t i = 0; i < high - 1; i++) {
    inds[i] = low + i;
    vals[i] = hh;
  }
  VecSetValues(b, high-low, inds, vals,
               INSERT_VALUES);
  free(inds); free(vals);
  \end{lstlisting}
  Similar but more involved for the matrix.
\end{frame}

\begin{frame}
  \frametitle{Declare the sparsity pattern}
  \begin{itemize}
  \item Why is this important?
  \item PETSc uses a popular sparse matrix data structure called
    \emph{compressed row storage} (CRS).
  \item Each nonzero element in the matrix is stored along with its column index
    in a single, one-dimensional array.
  \item Another array designates the start of each row.
  \end{itemize}
\end{frame}

\begin{frame}[fragile]
  \frametitle{Compressed row storage: example}
  \[
    \bm A = \begin{pmatrix}
      1 & 0 & 0 & 2 \\ 0 & 0 & 4 & 0 \\ 0 & 5 & 0 & 6 \\ 7 & 0 & 0 & 0
    \end{pmatrix}
  \]
  \begin{lstlisting}
    vals = [1, 2, 4, 5, 6, 7]
    cols = [0, 3, 2, 1, 3, 0]
    rowptrs = [0, 2, 3, 5]
  \end{lstlisting}
\end{frame}

\begin{frame}
  \frametitle{Compressed row storage}
  \begin{itemize}
  \item The problem with CRS is that inserting a new nonzero element somewhere
    in the matrix may involve a lot of data shifting.
  \item Other sparse structures exist that are optimized for insertion but not
    for matrix-vector operations.
  \item Various workarounds exist, such as assembling the final CSR structure
    after all elements have been set, or (in this case) pre-declare the sparsity
    pattern before inserting values.
  \item It is necessary to know how many nonzero elements are on each row. For
    MPI applications, it is also good to know how many nonzero elements are off
    or on the diagonal.
  \end{itemize}
\end{frame}

\begin{frame}[fragile]
  \frametitle{Declare the sparsity pattern}
  \begin{lstlisting}[style=c, basicstyle=\ttfamily\scriptsize]
    PetscInt first, last;
    MatGetOwnershipRange(A, &first, &last);
    PetscInt d_nz = (PetscInt *)
      malloc((last - first) * sizeof(PetscInt));
    PetscInt o_nz = (PetscInt *)
      malloc((last - first) * sizeof(PetscInt));

    for (size_t i = first; i < last; i++) {
      d_nz[i - first] = 5;
      o_nz[i - first] = 5;
    }
    MatMPIAIJSetPreallocation(
      A, PETSC_DEFAULT, d_nz, PETSC_DEFAULT, o_nz
    );
  \end{lstlisting}
  Here we have slightly overallocated for the sake of simplicity.
\end{frame}

\begin{frame}[fragile]
  \frametitle{Fully declare the sparsity pattern}
  \begin{itemize}
  \item We can also pin the sparsity pattern completely.
    \begin{lstlisting}[style=c, basicstyle=\ttfamily\scriptsize]
PetscInt d_nz = (PetscInt *) malloc(m*m * sizeof(PetscInt));
int total = 0;
// count number of nonzeros per row -> d_nz and total

MatSeqAIJSetPreallocation(A, PETSC_DEFAULT, d_nz);

PetscInt col = (PetscInt *) malloc(total * sizeof(PetscInt));
// compute the actual column indices
MatSeqAIJSetColumnIndices(A, col);
    \end{lstlisting}
  \item This way the matrix format will never change when adding values, which
    allows for multi-threaded assembly.
  \item Unfortunately it doesn't work in hybrid mode. (There is no
    \texttt{MatMPIAIJSetColumnIndices}.)
  \end{itemize}
\end{frame}

\begin{frame}
  \frametitle{Fill only your own elements}
  \begin{itemize}
  \item We can use \texttt{VecGetOwnershipRange} and
    \texttt{MatGetOwnershipRange} to get the global indices that are assigned to
    our own process.
  \item Then, each process can set just those indices.
  \item Note that in PETSc, a process owns whole \emph{rows} of the matrix, not
    columns as we have been (often) using in this course.
  \end{itemize}
\end{frame}

\begin{frame}[fragile]
  \frametitle{Out of tree builds}
  \begin{itemize}
  \item After celebrating out-of-tree builds, here is an actual example where
    they are useful.
  \item We want to compile both in serial and in parallel to compare the
    results. Therefore, make two build folders. For example:
    \begin{lstlisting}
mkdir serial && cd serial
PETSC_ARCH=linux-gnu-cxx-opt cmake ..
make

cd .. && mkdir parallel && cd parallel
PETSC_ARCH=linux-gnu-cxx-opt-mpi cmake ..
    -DENABLE_MPI=1
make
    \end{lstlisting}
  \end{itemize}
\end{frame}

\begin{frame}[fragile]
  \frametitle{Command-line arguments}
  \begin{itemize}
  \item PETSc has a set of built-in command-line arguments that all programs
    using PETSc will respect.
  \item These are read by functions with names like \texttt{XyzSetFromOptions}.
    We have used a few of them previously in this lecture.
  \item For instance, if we want to run our code using a Jacobi preconditioner
    and a GMRES solver, we can do
    \begin{lstlisting}
./poisson-petsc-smart 128 -pc_type jacobi
                          -ksp_type gmres
    \end{lstlisting}
  \end{itemize}
\end{frame}

\input{postamble}

\documentclass{beamer}

\usetheme{CambridgeUS}
\usecolortheme{orchid}

\usepackage[utf8]{inputenc}
\usepackage[T1]{fontenc}

% Paths
\newcommand{\figs}{../figs}
\newcommand{\data}{../data}
\newcommand{\code}{../code}

% URL styles
\usepackage{url}
\urlstyle{sf}

% Units
\usepackage[detect-weight=true, binary-units=true]{siunitx}
\DeclareSIUnit\flop{Flops}

% Math
\usepackage{amsmath}
\usepackage{amssymb}
\usepackage{bm}
\usepackage{nicefrac}
\newcommand{\dif}[1]{{\;\text{d}#1}}

% Graphics
\usepackage{graphicx}
\usepackage{caption}
\usepackage{subcaption}
\graphicspath{{../figs/}}

% Tikz
\usepackage{tikz}
\usetikzlibrary{positioning,shapes,arrows,calc,intersections}
\usepackage{pgfplots}
\usepgfplotslibrary{dateplot}
\pgfplotsset{compat=1.8}

% Colors
\definecolor{darkblue}{HTML}{00688B}
\definecolor{darkgreen}{HTML}{6E8B3D}
\definecolor{cadet}{HTML}{DAE1FF}
\definecolor{salmon}{HTML}{FFB08A}

% Listings
\usepackage{textcomp}
\usepackage{listings}
\lstset{
  keywordstyle=\bfseries\color{orange},
  stringstyle=\color{darkblue!80},
  commentstyle=\color{darkblue!80},
  showstringspaces=false,
  basicstyle=\ttfamily,
  upquote=true,
}
\lstdefinestyle{fortran}{
  language=Fortran,
  morekeywords={for},
  deletekeywords={status},
}
\lstdefinestyle{c}{
  language=C,
  morekeywords={include},
}
\lstdefinestyle{glsl}{
  language=C,
  morekeywords={attribute, vec2, vec3, vec4, varying, uniform, mat2, mat3, mat4},
}
\lstdefinestyle{cuda}{
  language=C,
  morekeywords={__global__, __device__, __host__},
}
\lstdefinestyle{shell}{
  language=bash,
  morekeywords={mkdir, ssh, cmake},
}

% Double hlines
\usepackage{hhline}

% Misc
\usepackage{nth}

\subtitle{TMA4280---Introduction to Supercomputing}

\begin{document}


\title{The Common library}
\author{Eivind Fonn}
\institute{SINTEF ICT / NTNU}
\date{December 2015}
\maketitle

\begin{frame}[fragile]
  \frametitle{How to build}
  Easiest to use CMake.
\begin{lstlisting}
  add_subdirectory(../common common)
  list(APPEND CMAKE_C_FLAGS ${COMMON_C_FLAGS})
  include_directories(${COMMON_INCLUDE_PATHS})
  target_link_libraries(myprog ${COMMON_LIBRARIES})
\end{lstlisting}
%$
  See the \texttt{sum} example for how this is done.
\end{frame}

\begin{frame}[fragile]
  \frametitle{Initializing}
\begin{lstlisting}[style=c]
  void init_app(int argc, char **argv,
                int *rank, int *size);
\end{lstlisting}
  Takes care of initializing MPI if necessary, and asking about the MPI
  threading level. You should call this instead of \texttt{MPI\_Init},
  \texttt{MPI\_Comm\_size} and \texttt{MPI\_Comm\_rank}, because the common
  library also needs to know these things.
\end{frame}

\begin{frame}[fragile]
  \frametitle{Vector}
\begin{lstlisting}[style=c]
  typedef struct {
    double* data;
    int len;
    int glob_len;
    int stride;
    MPI_Comm* comm;
    int comm_size;
    int comm_rank;
    int* displ;
    int* sizes;
  } vector_t;

  typedef vector_t* Vector;
\end{lstlisting}
\end{frame}

\begin{frame}[fragile]
  \frametitle{Vector}
  \begin{itemize}
  \item \texttt{data} is a pointer to the contiguous vector data.
  \item \texttt{len} is the local length of the vector (how many elements are
    stored in \texttt{data}).
  \item \texttt{glob\_len} is the global length of the vector (over all
    processes).
  \item \texttt{stride} is the distance between each element in memory.
  \item \texttt{comm} is the MPI communicator across which the vector is split
    (if applicable).
  \item \texttt{comm\_size} and \texttt{comm\_rank} store the number of
    processors and the rank of this process.
  \item \texttt{displ} and \texttt{sizes} store the offset and sizes of each
    part of the vector.
  \end{itemize}
\end{frame}

\begin{frame}[fragile]
  \frametitle{Making vectors}
\begin{lstlisting}[style=c]
  Vector createVector(int len);
\end{lstlisting}
  Creates a serial vector. That means
  \begin{itemize}
  \item \texttt{len == glob\_len}
  \item \texttt{stride == 1}
  \item \texttt{comm == displ == sizes == NULL}
  \item \texttt{comm\_size == 1}
  \item \texttt{comm\_rank == 0}
  \end{itemize}
\end{frame}

\begin{frame}[fragile]
  \frametitle{Making vectors}
\begin{lstlisting}[style=c]
  Vector createVectorMPI(int glob_len, MPI_Comm *,
                         int allocdata, int pad);
\end{lstlisting}
  \begin{itemize}
  \item Creates a parallel vector. It is ``automagically'' split between the
    processes in the communicator, and the \texttt{displ} and \texttt{sizes}
    attributes should reflect this. \texttt{len} may not be equal to
    \texttt{glob\_len}, but \texttt{stride} is still equal to one.
  \item \texttt{allocdata = 0} will not allocate the data itself.
  \item \texttt{pad = 1} will pad the vector with an additional element at the
    beginning and the end, except at the ``boundary'' processes. Useful for
    tridiagonal matrix-vector products.
  \end{itemize}
\end{frame}

\begin{frame}[fragile]
  \frametitle{Matrix}
\begin{lstlisting}[style=c]
  typedef struct {
    double** data;
    Vector as_vec;
    Vector* col;
    Vector* row;
    int rows;
    int cols;
    int glob_rows;
    int glob_cols;
  } matrix_t;

  typedef matrix_t* Matrix;
\end{lstlisting}
\end{frame}

\begin{frame}[fragile]
  \frametitle{Matrix}
  \begin{itemize}
  \item \texttt{}
  \item \texttt{data} is a pointer to the contiguous vector data.
  \item \texttt{len} is the local length of the vector (how many elements are
    stored in \texttt{data}).
  \item \texttt{glob\_len} is the global length of the vector (over all
    processes).
  \item \texttt{stride} is the distance between each element in memory.
  \item \texttt{comm} is the MPI communicator across which the vector is split
    (if applicable).
  \item \texttt{comm\_size} and \texttt{comm\_rank} store the number of
    processors and the rank of this process.
  \item \texttt{displ} and \texttt{sizes} store the offset and sizes of each
    part of the vector.
  \end{itemize}
\end{frame}

\begin{frame}[fragile]
  \frametitle{Types}
  \begin{itemize}
  \item Ints and chars optionally can be \emph{signed} or \emph{unsigned}. A
    signed int can (for example) take values in $[-2^{31}, 2^{31})$ while an
    unsigned one can take values in $[0,2^{32})$.
  \item The following code will, on most computers, print ``$200, -56$''.
\begin{lstlisting}[style=c]
  unsigned char a = 200;
  signed char b = 200;
  printf("%d, %d\n", a, b);
\end{lstlisting}
  \item Signed is implied if not otherwise specified.
  \end{itemize}
\end{frame}

\begin{frame}[fragile]
  \frametitle{Types}
  \begin{itemize}
  \item Ints can additionally be either short, normal, long or long long.
  \item Their lengths are guaranteed to be non-decreasing. Shorts and normal
    ints are at least 16 bits, long ints are at least 32 bits and long long ints
    are at least 64 bits.
  \item These qualifiers can be mixed with signed or unsigned. The typename
    \texttt{int} can also be omitted. These four variables have identical types:
\begin{lstlisting}[style=c]
  long a;
  signed long b;
  long int c;
  signed long int d;
\end{lstlisting}
  \end{itemize}
\end{frame}

\begin{frame}
  \frametitle{Other types}
  \begin{itemize}
  \item \texttt{size\_t} is an unsigned integer type used for \emph{sizes} of
    things in memory. For example, \texttt{malloc} expects a \texttt{size\_t}
    parameter.
  \item \texttt{ptrdiff\_t} is a signed integer type used for differences
    between memory locations.
  \item Both of these are useful for indexing arrays. Where only nonnegative
    indices are expected (which is almost always the case), I prefer
    \texttt{size\_t}.
  \end{itemize}
\end{frame}

\begin{frame}
  \frametitle{Other types}
  Other types with predictable and portable sizes were introduced in C99: Here,
  \texttt{N} can be $8$, $16$, $32$ or $64$.
  \begin{itemize}
  \item \texttt{[u]intN\_t}: integer type with exact width.
  \item \texttt{[u]int\_leastN\_t}: integer type with at least a given width.
  \item \texttt{[u]int\_fastN\_t}: the \emph{fastest} type on the given machine
    with at least a given width.
  \end{itemize}
\end{frame}

\begin{frame}[fragile]
  \frametitle{Arrays}
  \begin{itemize}
  \item Arrays can be declared and initialized
\begin{lstlisting}[style=c]
  int cat[4];  // Uninitialized
  int cat[4] = {1, 2, 3, 4};
  int cat[] = {1, 2, 3, 4};
  int cat[];  // Error
\end{lstlisting}
  \item Since C99, the array size may be known only at runtime. This is an error
    in ANSI C:
\begin{lstlisting}[style=c]
  int size = 4;
  int cat[size];  // Uninitialized
\end{lstlisting}
  \end{itemize}
\end{frame}

\begin{frame}[fragile]
  \frametitle{Arrays}
  \begin{itemize}
  \item Multidimensional arrays are stored contiguously in memory, in row-major
    order.
  \item For this reason, you can only leave the first dimension unspecified.
    Otherwise the compiler doesn't know how to compute the memory location of an
    array element from its index.
\begin{lstlisting}[style=c]
  int cat[4][4];
  int cat[4][4] = { { ... }, { ... }, ... };
  int cat[][4] = { { ... }, { ... }, ... };
  int cat[][] = ...;  // Error
  int cat[][4];  // Error
\end{lstlisting}
  \end{itemize}
\end{frame}

\begin{frame}[fragile]
  \frametitle{Pointers}
  \begin{itemize}
  \item Pointers are memory addresses. Encoded in the type of a pointer is also
    the type of the thing it points to.
\begin{lstlisting}[style=c]
  int *a;  // Pointer to int
  double *b;  // Pointer to double
  float **c;  // Pointer to pointer to float
\end{lstlisting}
  \item To get the value that a pointer points to, use dereferencing with
    \texttt{*}. To get the address of a variable, use \texttt{\&}:
\begin{lstlisting}[style=c]
  int a = 10;
  int *b = &a;
  printf("b points to the value %d\n", *b);
  printf("a has address %x\n", &a);
\end{lstlisting}
  \end{itemize}
\end{frame}

\begin{frame}
  \frametitle{Pointers}
  \begin{itemize}
  \item It is easy to confuse the \texttt{*} in the declaration and the
    dereferencing operator.
  \item Helpful way of thinking: \texttt{int *a;} declares \texttt{*a} as an
    \texttt{int}. This is why C people like to put the \texttt{*} next to the
    variable name instead of the type.
  \item \textbf{Note:} this is just a mnemonic device. It does not actually
    declare an int.
  \end{itemize}
\end{frame}

\begin{frame}
  \frametitle{Pointers}
  \begin{itemize}
  \item The warning about not using uninitialized memory is extremely important
    for pointers.
  \item There is no guarantee that an uninitialized pointer points to
    somewhere that makes sense, or is even a legal memory location.
  \item Dereferencing such a pointer may cause your program to crash
    (segmentation violation).
  \item It is common to use \texttt{0} or \texttt{NULL} as a value to indicate
    an illegal pointer.
  \item Functions that return pointers may use \texttt{NULL} to indicate an
    error.
  \item Functions that accept pointer arguments may interpret \texttt{NULL} as
    the absence of a value.
  \item \textbf{Be careful!} The only correct answer is also a bad one: check
    every pointer everywhere.
  \end{itemize}
\end{frame}

\begin{frame}[fragile]
  \frametitle{Pointers as arrays}
  \begin{itemize}
  \item Pointers are unsigned integers and can be subject to arithmetic.
  \item The math is always in terms of the underlying type.
\begin{lstlisting}[style=c]
  int a = 10;
  int *b = &a + 1;
\end{lstlisting}
    The pointer \texttt{b} now points to a memory location \emph{four bytes}
    offset from where \texttt{a} is stored (if ints are four bytes).
  \item Therefore we can use pointers as a form of arrays.
\begin{lstlisting}[style=c]
  int *ptr = ...;
  *ptr;        // First element
  *(ptr + 1);  // Second element
  *(ptr + 2);  // Third element
\end{lstlisting}
  \end{itemize}
\end{frame}

\begin{frame}[fragile]
  \frametitle{Pointers as arrays}
  \begin{itemize}
  \item This is so useful that C allows us to use \texttt{[]}-indexing on
    pointers.
  \item If \texttt{p} has a pointer type then \texttt{p[n]} is
    \emph{exactly equivalent} to \texttt{*(p+n)}.
  \item Interestingly, since \texttt{p+n} is the same as \texttt{n+p}, you can
    also use \texttt{n[p]}.
  \item This has caused a lot of confusion about the difference between pointers
    and arrays.
  \end{itemize}
\end{frame}

\begin{frame}[fragile]
  \frametitle{Pointers as arrays (1D)}
\begin{lstlisting}[style=c]
  int *ptr;
  int arr[];
  ptr[5]; arr[5];
\end{lstlisting}
  \begin{itemize}
  \item For \emph{one-dimensional} arrays, array indexing and pointer indexing
    work pretty much exactly the same way.
  \item Array indexing may be slightly faster since it involves less redirection
    (For arrays, the program gets ``for free'' the location of the first
    element. In the pointer case, the location of the first element is something
    that's stored in a variable which has to be looked up.)
  \item When passed to a function, \textbf{arrays become pointers}.
  \end{itemize}
\end{frame}

\begin{frame}[fragile]
  \frametitle{Pointers as arrays (1D)}
\begin{lstlisting}[style=c]
  int **ptr;
  int arr[][8];
  ptr[5][3]; arr[5][3];
\end{lstlisting}
  \begin{itemize}
  \item For \emph{multi-dimensional} arrays, they are very different things.
  \item Array: get the element at offset $5\times8 + 3 = 43$ from the start.
  \item Pointer: Follow the pointer, go to offset $5$, follow \emph{that}
    pointer, go to offset $3$.
  \item When passing a multi-dimensional array to a function, only the ``top''
    layer becomes a pointer. The other layers of redirection can't be
    automatically created.
  \end{itemize}
\end{frame}

\begin{frame}[fragile]
  \frametitle{Strings}
  \begin{itemize}
  \item C notably does not have a boolean type.
  \item Instead it uses integers. Anything that is not zero is true.
  \item Pitfall:
\begin{lstlisting}[style=c]
  unsigned char b = 256;
  if (b) { ... };
\end{lstlisting}
  \end{itemize}
\end{frame}

\begin{frame}[fragile]
  \frametitle{Strings}
  \begin{itemize}
  \item Also notably, C does not have a string type.
  \item Instead, we use pointers to chars.
  \item A C string implicitly keeps going until a zero byte is found. This byte
    is not part of the string but it is part of the allocated memory for it.
  \item The meaning of the \texttt{char **argv} argument to \texttt{main} is
    then to be understood as an ``array'' of ``strings''.
  \item \texttt{strcmp} compares two strings. It returns zero if they are equal.
    Therefore it should be negated in an \texttt{if}.
  \end{itemize}
\end{frame}

\begin{frame}[fragile]
  \frametitle{Structs}
  \begin{itemize}
  \item A struct is a handy way of grouping variables together.
\begin{lstlisting}[style=c]
  struct person {
    int age;
    char *name, *address;
    double height, weight;
  };

  struct person me;
  me.name = "Eivind Fonn";
  me.age = 31;
\end{lstlisting}
  \end{itemize}
\end{frame}

\begin{frame}[fragile]
  \frametitle{Structs}
  \begin{itemize}
  \item We often find ourselves with pointers to structs.
  \item This is ugly:
\begin{lstlisting}[style=c]
  struct person *me;
  (*me).age;
\end{lstlisting}
  \item Instead, use \texttt{->} as a shortcut.
\begin{lstlisting}[style=c]
  struct person *me;
  me->age;
\end{lstlisting}
  \end{itemize}
\end{frame}

\begin{frame}[fragile]
  \frametitle{Functions}
  \begin{itemize}
  \item Look very similar to functions in other C-like languages.
\begin{lstlisting}[style=c]
  <rtype> <name>(<arguments>)
  {
    <code>
    return <expr>;
  }
\end{lstlisting}
  \item Arguments are always passed by value. If you want to pass by reference,
    to modify the arguments in a way that is visible to the caller, you must use
    pointers.
  \item This is a handy way to return more than one value.
  \end{itemize}
\end{frame}

\begin{frame}[fragile]
  \frametitle{Functions}
  For example:
\begin{lstlisting}[style=c]
  int sum(int a, int b) {
    return a + b;
  }
  int product(int a, int b) {
    return a * b;
  }
  void sum_and_product(int a, int b,
                       int *sum, int *product)
  {
    *sum = a + b;
    *product = a * b;
  }
\end{lstlisting}
\end{frame}

\begin{frame}[fragile]
  \frametitle{Functions}
  \begin{itemize}
  \item A function must be \emph{declared} before it is \emph{used} in a source
    code file. The declaration includes all information the compiler needs to
    use the function:
\begin{lstlisting}[style=c]
    int sum(int a, int b);
    int sum(int, int);
\end{lstlisting}
  \item As long as you have provided a \emph{definition} when the program is
    linked, it's okay.
  \item This lets us split up code into different source code files, so long as
    we keep the declarations separate (in header files).
  \item When you \texttt{\#include} files, you're importing a bunch of function
    declarations.
  \end{itemize}
\end{frame}

\begin{frame}[fragile]
  \frametitle{Arrays, pointers and functions}
\begin{lstlisting}[style=c]
  void fncA(int *a);
  void fncB(int b[]);  // Equivalent to fncA

  void fncC(int **c);
  void fncD(int d[][8]);  // NOT equivalent to fncC
  void fncE(int (*e)[8]);  // Equivalent to fncD

  int arr_1d[10];
  fncA(arr_1d); fncB(arr_1d);  // OK

  int arr_2d[10][8];
  fncC(arr_2d);  // Not OK
  fncD(arr_2d); fncE(arr_2d);  // OK
\end{lstlisting}
\end{frame}

\begin{frame}[fragile]
  \frametitle{Function pointers}
  \begin{itemize}
  \item You can store a pointer to a function.
  \item This declares \texttt{func\_ptr} as a pointer to a function that accepts
    a char and a pointer to a double, and returns an int. Then it sets this
    pointer to the address of the function \texttt{actual\_func}.
\begin{lstlisting}[style=c]
  int (*func_ptr)(char, double*) = &actual_func;
\end{lstlisting}
  \item You can call a function pointer by dereferencing it.
\begin{lstlisting}[style=c]
  int value = (*func_ptr)('a', &some_double);
\end{lstlisting}
  \end{itemize}
\end{frame}

\begin{frame}[fragile]
  \frametitle{Typedefs}
  \begin{itemize}
  \item C types can be famously complicated to figure out.
\begin{lstlisting}[style=c]
  void (*f)(int,void (*)());
  int **(*f)(int**,int**(*)(int**,int**));
  char (*(*X[3])())[5];
\end{lstlisting}
  \item Helpful tool: cdecl
\begin{lstlisting}
  cdecl> explain void (*f)(int,void(*)())
  declare f as pointer to function
  (int, pointer to function returning void)
  returning void
\end{lstlisting}
  \end{itemize}
\end{frame}

\begin{frame}[fragile]
  \frametitle{Typedefs}
  \begin{itemize}
  \item Better technique: declare type aliases with \texttt{typedef}.
  \item Very useful for structs so you don't have to type \texttt{struct person}
    every time.
\begin{lstlisting}[style=c]
  typedef struct person person_t;
\end{lstlisting}
  \item You can typedef together with the declaration too.
\begin{lstlisting}[style=c]
  typedef struct {
    int age;
    char *name, *address;
    double height, weight;
  } person_t;
\end{lstlisting}
  \item Also useful for types that will always be pointers.
\begin{lstlisting}[style=c]
  typedef person_t* personPtr;
\end{lstlisting}
  \end{itemize}
\end{frame}

\begin{frame}[fragile]
  \frametitle{Constants}
  \begin{itemize}
  \item A variable can be declared \texttt{const} to indicate that its value
    will not change.
  \item In function arguments, \texttt{const} can be used with pointers to
    indicate that the function will not change the data it points to.
  \item The latter allows compiler optimization, so use it when possible!
  \item Easy to get confused, though:
\begin{lstlisting}[style=c]
  int const *ptr;  // *ptr is constant
  int * const ptr;  // ptr is constant
  int const * const ptr;  // both
\end{lstlisting}
  \end{itemize}
\end{frame}

\begin{frame}[fragile]
  \frametitle{Preprocessor directives}
  \begin{itemize}
  \item You can use \texttt{\#define} to define symbolic constants. These are
    substituted textually before compilation, so they're effectively the same as
    writing a literal value in your program:
\begin{lstlisting}[style=c]
  #define PI 3.141516;
  return r * PI * PI;

  // turns into
  return r * 3.141516 * 3.141516;
\end{lstlisting}
  \item Note that they are not variables, not even const variables.
\begin{lstlisting}[style=c]
  &PI;  // Error
\end{lstlisting}
  \end{itemize}
\end{frame}

\begin{frame}
  \frametitle{Miscellaneous tips}
  \begin{itemize}
  \item \texttt{malloc} and \texttt{calloc} allocate memory and return pointers.
    The latter zeroes the allocated memory, which makes it safer but also
    slower.
  \item \texttt{sizeof} is a compile-time macro for getting the size of a type.
    It is useful together with \texttt{malloc} and \texttt{calloc}.
  \item If you allocate memory but ``lose'' the pointer to it, that memory
    cannot be used by anything else until your program exits. Use \texttt{free}
    to release memory you don't need any more.
  \item Use \texttt{atoi} and \texttt{atof} to convert strings (e.g.
    command-line arguments) to integers or doubles.
  \item Use \texttt{printf} to convert to string (and output).
  \item Use casting for other automatic conversions: \texttt{(double)some\_int}.
  \item Just because it's easy to mess up in C doesn't mean you should litter
    your program with safety checks. We're not NASA.
  \end{itemize}
\end{frame}

\begin{frame}[fragile]
  \frametitle{Miscellaneous tips}
  \begin{itemize}
  \item Store multidimensional arrays contiguously. This lets us use BLAS. If
    you \texttt{malloc} inside a loop you may be doing it wrong. You can
    construct an array of pointers afterwards, if you like.
\begin{lstlisting}[style=c]
  double *data =
    malloc(M * N * sizeof(double));
  double *matrix[N];
  for (size_t i = 0; i < N; i++) {
    matrix[i] = &data[i*M];
    // or
    matrix[i] = data + i*M;
  }
\end{lstlisting}
  \end{itemize}
\end{frame}

\end{document}


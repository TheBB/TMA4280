\documentclass{beamer}

\usetheme{CambridgeUS}
\usecolortheme{orchid}

\usepackage[utf8]{inputenc}
\usepackage[T1]{fontenc}

% Paths
\newcommand{\figs}{../figs}
\newcommand{\data}{../data}
\newcommand{\code}{../code}

% URL styles
\usepackage{url}
\urlstyle{sf}

% Units
\usepackage[detect-weight=true, binary-units=true]{siunitx}
\DeclareSIUnit\flop{Flops}

% Math
\usepackage{amsmath}
\usepackage{amssymb}
\usepackage{bm}
\usepackage{nicefrac}
\newcommand{\dif}[1]{{\;\text{d}#1}}

% Graphics
\usepackage{graphicx}
\usepackage{caption}
\usepackage{subcaption}
\graphicspath{{../figs/}}

% Tikz
\usepackage{tikz}
\usetikzlibrary{positioning,shapes,arrows,calc,intersections}
\usepackage{pgfplots}
\usepgfplotslibrary{dateplot}
\pgfplotsset{compat=1.8}

% Colors
\definecolor{darkblue}{HTML}{00688B}
\definecolor{darkgreen}{HTML}{6E8B3D}
\definecolor{cadet}{HTML}{DAE1FF}
\definecolor{salmon}{HTML}{FFB08A}

% Listings
\usepackage{textcomp}
\usepackage{listings}
\lstset{
  keywordstyle=\bfseries\color{orange},
  stringstyle=\color{darkblue!80},
  commentstyle=\color{darkblue!80},
  showstringspaces=false,
  basicstyle=\ttfamily,
  upquote=true,
}
\lstdefinestyle{fortran}{
  language=Fortran,
  morekeywords={for},
  deletekeywords={status},
}
\lstdefinestyle{c}{
  language=C,
  morekeywords={include},
}
\lstdefinestyle{glsl}{
  language=C,
  morekeywords={attribute, vec2, vec3, vec4, varying, uniform, mat2, mat3, mat4},
}
\lstdefinestyle{cuda}{
  language=C,
  morekeywords={__global__, __device__, __host__},
}
\lstdefinestyle{shell}{
  language=bash,
  morekeywords={mkdir, ssh, cmake},
}

% Double hlines
\usepackage{hhline}

% Misc
\usepackage{nth}

\subtitle{TMA4280---Introduction to Supercomputing}

\begin{document}


\title{Living in the shell}
\author{Eivind Fonn}
\institute{SINTEF ICT / NTNU}
\date{December 2015}
\maketitle

\begin{frame}
  \frametitle{The shell - your best friend}

  \begin{itemize}
  \item Computers are commonly operated using graphical user interfaces.
  \item Supercomputers are not common computers.
  \item Some systems have web interfaces for dispatching jobs - ``cloud computing''.
  \item These are only frontends to a shell based system.
  \item Almost all (probably 99\%) of these systems are based on some Unix (mostly Linux today).
  \item Many different shells exist: csh, tcsh, sh, \ldots.
  \item The most common shell on Linux is the Bourne Again SHell - \emph{bash}.
  \end{itemize}
\end{frame}

\begin{frame}
  \frametitle{Bash}

  \begin{itemize}
  \item The (bash) shell is very powerful: built-in scripting.
  \item Scripts used for jobs on the supercomputer facilities.
  \item Basic shell usage will be covered.
  \item Some basic utilities available on the shell will be covered.
  \item Compiling software on the shell will be covered, by hand as well as automated.
  \item Some basics of shell scripting will be covered.
  \end{itemize}
\end{frame}

\begin{frame}
  \frametitle{Basic navigation}

  \begin{itemize}
    \setlength{\itemindent}{1cm}
  \item[ls] List the files in a directory.
  \item[cd] Navigate around in the directory structure. (\texttt{cd ..})
  \item[mkdir] Create a new directory.
  \item[rm] Delete a file or directory.
  \item[mv] Move or rename a file or directory.
  \end{itemize}
\end{frame}

\begin{frame}
  \frametitle{Long live laziness}

  \begin{itemize}
  \item All commands have readily available documentation.
  \item \texttt{man foo} or \texttt{info foo} documents parameters and explains
    usage scenarios.
  \item Bash has \emph{tab completion}: whenever a command/file/directory is
    qualified hit tab to have bash complete it to lessen the amount of typing
    required.
  \item Use the \emph{arrow keys} to recall previously executed commands.
  \item Use \emph{Ctrl-r} to search in previously executed commands (anywhere,
    not just the start).
  \end{itemize}
\end{frame}

\begin{frame}
  \frametitle{Handy utilities}
  \begin{itemize}
    \setlength{\itemindent}{1cm}
  \item[cat] Show the contents of a file. \\ \texttt{cat test.c}
  \item[less] Show the contents of a file with pagination. \\ \texttt{less test.c}
  \item[find] Locate a lost file. \\ \texttt{find .~-name "*.c"}
  \item[grep] Search for text in a file. \\ \texttt{grep "foo" test.c}
  \item[sed] Search and replace text in a file. \\ \texttt{sed -e 's/foo/bar/g' -i test.c}
  \end{itemize}
\end{frame}

\begin{frame}
  \frametitle{Output redirection}
  \begin{itemize}
  \item Sometimes we want to combine several utilities, in the sense that the output from
    one command is the input to another one.
  \item This can be achieved through \emph{pipes}. \\
    \texttt{cat test.c | grep "foo" | awk -F ' ' '\{print\ \$2\}'}
  \item Sometimes we want to redirect the output of a program to a file. Use
    \texttt{>} to do this. \\
    \texttt{grep "foo" test.c > output.txt}
  \item You can append to the output file using \texttt{>\hspace{0pt}>}. \\
    \texttt{grep "bar" test.c >\hspace{0pt}> output.txt}
  \end{itemize}
\end{frame}

\begin{frame}
  \frametitle{compiling software}
  \begin{itemize}
  \item Software is compiled by invoking a compiler, and finally a linker, on the shell.
  \item Compiler commands vary across systems.
  \item Compiler commands depend on the language used.
  \item However, there are a lot of similarities as well.
  \item We here consider Linux systems using the GNU compilers.
  \item The basic approach is:
    \begin{itemize}
    \item Compile sources into object (.o) files.
    \item Link objects into a library (.a/.so) or an executable.
    \end{itemize}
  \end{itemize}
\end{frame}

\begin{frame}[fragile]
  \frametitle{Our first C program}
  We first consider compiling a simple hello world program.
  \begin{center}
    \begin{tabular}{c}
      \lstinputlisting[style=c]{\code/hello/single/hello.c}
    \end{tabular}
  \end{center}
\end{frame}

\begin{frame}
  \frametitle{Our first C program}
  \begin{itemize}
  \item We only have a single source file.
  \item We can then do the compilation and linking in a single command. \\
    \texttt{gcc -o hello hello.c}
  \item We might want to turn on compiler optimizations. \\
    \texttt{gcc -o hello hello.c -O2}
  \item We might want to include debugging info. \\
    \texttt{gcc -o hello hello.c -g}
  \end{itemize}
\end{frame}

\begin{frame}[fragile]
  \frametitle{Our first Fortran program}
  The corresponding Fortran format looks like this.
  \begin{center}
    \begin{tabular}{c}
      \lstinputlisting[style=fortran]{\code/hello/fortran/hello.f90}
    \end{tabular}
  \end{center}
\end{frame}

\begin{frame}
  \frametitle{Our first Fortran program}
  \begin{itemize}
    \item We only have a single source file.
    \item If you do not name your file .f90, then you must use
      \texttt{-free-form} (or you have to write Fortran 77 code, and nobody
      wants that.)
    \item We can then do the compilation and linking in a single command. \\
      \texttt{gfortran -o hello hello.f}
    \item We might want to turn on compiler optimizations. \\
      \texttt{gfortran -o hello hello.f -O2}
    \item We might want to include debugging info. \\
      \texttt{gfortran -o hello hello.f -g}
  \end{itemize}
\end{frame}

\begin{frame}[fragile]
  \frametitle{Our second C program}
  It is impractical and unadvisable to have all sources of a complex program in
  a single source file. We here consider a more involved hello world program. We
  put the printing in a separate source file.
  \begin{center}
    \begin{tabular}{c}
      \lstinputlisting[style=c]{\code/hello/multiple/utils.c}
    \end{tabular}
  \end{center}
\end{frame}

\begin{frame}[fragile]
  \frametitle{Our second C program}
  We have to write a \emph{header file} so we can use these functions in another
  source file.
  \begin{center}
    \begin{tabular}{c}
      \lstinputlisting[style=c]{\code/hello/multiple/utils.h}
    \end{tabular}
  \end{center}
\end{frame}

\begin{frame}[fragile]
  \frametitle{Our second C program}
  Finally, we have the main program.
  \begin{center}
    \begin{tabular}{c}
      \scalebox{0.6}{\lstinputlisting[style=c]{\code/hello/multiple/hello.c}}
    \end{tabular}
  \end{center}
\end{frame}

\begin{frame}[fragile]
  \frametitle{Our second C program}
  \begin{itemize}
  \item In theory we can compile in one step as we did for the first program. \\
    \texttt{gcc -o hello hello.c utils.c}
  \item The problem is that all sources are recompiled every time, even if
    nothing changed in some of them.
  \item Better approach: First sources are compiled into objects, then objects
    are linked together. \\
\begin{lstlisting}[style=shell]
gcc -c -o hello.o hello.c
gcc -c -o utils.o utils.c
gcc -o hello hello.o utils.o
\end{lstlisting}
  \end{itemize}
\end{frame}

\begin{frame}
  \frametitle{Our third C program}
  Often we want to share code between programs. The most convenient way to do
  that is to link them into a library, which can be reused.

  We have a function for printing hello:
  \begin{center}
    \begin{tabular}{c}
      \lstinputlisting[style=c]{\code/hello/library/hello_utils.c}
    \end{tabular}
  \end{center}
\end{frame}

\begin{frame}
  \frametitle{Our third C program}
  And one for printing goodbye:
  \begin{center}
    \begin{tabular}{c}
      \lstinputlisting[style=c]{\code/hello/library/goodbye_utils.c}
    \end{tabular}
  \end{center}
\end{frame}

\begin{frame}
  \frametitle{Our third C program}
  The main program looks like this.
  \begin{center}
    \begin{tabular}{c}
      \scalebox{0.5}{\lstinputlisting[style=c]{\code/hello/library/main.c}}
    \end{tabular}
  \end{center}
\end{frame}

\begin{frame}[fragile]
  \frametitle{Our third C program}
  \begin{itemize}
  \item We first link the printing functions into a library
\begin{lstlisting}[style=shell]
gcc -c -o hello_utils.o hello_utils.c
gcc -c -o goodbye_utils.o goodbye_utils.c
ar r libutils.a hello_utils.o goodbye_utils.o
\end{lstlisting}
  \item Then we build the program.
\begin{lstlisting}[style=shell]
gcc -c -o main.o main.c
gcc -o hello main.o libutils.a
\end{lstlisting}
  \end{itemize}
\end{frame}

\begin{frame}
  \frametitle{Our third C program}
  We can write another main source file.
  \begin{center}
    \begin{tabular}{c}
      \scalebox{0.5}{\lstinputlisting[style=c]{\code/hello/library/main_reversed.c}}
    \end{tabular}
  \end{center}
\end{frame}

\begin{frame}[fragile]
  \frametitle{Our third C program}
  We can then compile it and link against the existing library.
\begin{lstlisting}[style=shell]
gcc -c -o main_reversed.o main_reversed.c
gcc -o hello_reversed main_reversed.o libutils.a
\end{lstlisting}
  Or we can use this (more common) form:
\begin{lstlisting}[style=shell]
gcc -o hello_reversed main_reversed.o -L. -lutils
\end{lstlisting}
\end{frame}

\begin{frame}
  \frametitle{Automating builds}
  \begin{itemize}
  \item Typing all these compilation, archiving and linking commands is
    tedious and error prone.
  \item There are tools to handle this - a common choice is the (GNU)
    \emph{make} system.
  \item We have to write a Makefile with instructions. These are text files
    with a particular syntax.
  \end{itemize}
\end{frame}

\begin{frame}
  \frametitle{Automating builds}
  A sample makefile for our third C program.
  \begin{center}
    \begin{tabular}{c}
      \scalebox{0.5}{\lstinputlisting{\code/hello/make/Makefile}}
    \end{tabular}
  \end{center}
\end{frame}

\begin{frame}
  \frametitle{Automating builds}
  \begin{itemize}
  \item Make does not take care of build dependencies, that is locating
    required libraries etc.
  \item Make does not take care of setting up compiler flags. These are often
    machine specific, thus hardcoding them is not a good idea.
  \item There exists other tools to handle this.
  \item Traditionally: GNU autotools has been used a lot.
  \item Very powerful, very confusing, steep learning curve.
  \end{itemize}
\end{frame}

\begin{frame}
  \frametitle{CMake}
  \begin{itemize}
  \item CMake is easier to grasp than autotools.
  \item The C stands for \emph{cross platform}, so the idea is that
    you generate build systems for the different platform based on a common
    script.
  \item CMake can generate Makefiles for unix, xcode projects for OSX and
    .vcxproj files for visual studio.
  \item Simplifies cross platform development since you do not have to manually
    maintain build systems.
  \end{itemize}
\end{frame}

\begin{frame}
  \frametitle{CMake}
  A sample CMakeLists.txt for our third C program.
  \begin{center}
    \begin{tabular}{c}
      \scalebox{0.8}{\lstinputlisting{\code/hello/cmake/CMakeLists.txt}}
    \end{tabular}
  \end{center}
\end{frame}

\begin{frame}[fragile]
  \frametitle{Out of tree builds}
  \begin{itemize}
  \item Often, it is useful to have several configurations of a program easily
    available.
  \item Examples of this can be with and without optimizations, with and without
    OpenMP etc.
  \item CMake makes this very easy, as you can do out-of-tree builds.
  \item This is achieved by
\begin{lstlisting}[style=shell]
mkdir build
cd build
cmake ..
\end{lstlisting}
  \end{itemize}
\end{frame}

\begin{frame}
  \frametitle{Shell scripting}
  \begin{itemize}
  \item A common task when developing/using scientific software is analyzing some
    output from a program, checking for correctness, performance etc.
  \item Modifying the sources for these tasks are often tedious and error prone.
  \item In fact, it can even be impossible if you are using software you do not
    have the sources for.
  \item Shell scripts are very useful for such tasks.
  \item Instead of modifying the program, we write a script which repeatedly
    executes the program and captures the interesting output.
  \end{itemize}
\end{frame}

\begin{frame}
  \frametitle{Shell scripting}
  \begin{itemize}
  \item We here consider a simple little program which, given a number of repetitions
    and a vector length performs the dotproduct of random vectors.
  \item The program prints the elapsed time to stdout.
  \item We want to analyze how the algorithm scales by repeatedly calling
    the program and catching the given runtime.
  \end{itemize}
\end{frame}

\begin{frame}
  \frametitle{Shell scripting}
  \begin{center}
    \begin{tabular}{c}
      \scalebox{0.6}{\lstinputlisting[style=c,firstline=29,lastline=47]{\code/timing/timing.c}}
    \end{tabular}
  \end{center}
\end{frame}

\begin{frame}
  \frametitle{Shell scripting}
  \begin{center}
    \begin{tabular}{c}
      \scalebox{0.6}{\lstinputlisting[style=shell]{\code/timing/timing.sh}}
    \end{tabular}
  \end{center}
\end{frame}

\end{document}


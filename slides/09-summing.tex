\documentclass{beamer}

\usetheme{CambridgeUS}
\usecolortheme{orchid}

\usepackage[utf8]{inputenc}
\usepackage[T1]{fontenc}

% Math
\usepackage{amsmath}
\usepackage{amssymb}
\usepackage{bm}

% Graphics
\usepackage{graphicx}
\usepackage{caption}
\usepackage{subcaption}
\graphicspath{{../figs/}}

% Colors
\definecolor{darkblue}{HTML}{00688B}
\definecolor{darkgreen}{HTML}{6E8B3D}
\definecolor{cadet}{HTML}{DAE1FF}
\definecolor{salmon}{HTML}{FFB08A}

% Listings
\usepackage{textcomp}
\usepackage{listings}
\lstset{
  keywordstyle=\bfseries\color{orange},
  stringstyle=\color{darkblue!80},
  commentstyle=\color{darkblue!80},
  showstringspaces=false,
  basicstyle=\ttfamily,
  upquote=true,
}
\lstdefinestyle{fortran}{
  language=Fortran,
  morekeywords={for},
  deletekeywords={status},
}
\lstdefinestyle{c}{
  language=C,
  morekeywords={include},
}
\lstdefinestyle{shell}{
  language=bash,
}

\subtitle{TMA4280---Introduction to Supercomputing}

\begin{document}


\title{Summing with OpenMP and MPI}
\author{Eivind Fonn}
\institute{SINTEF ICT / NTNU}
\date{December 2015}
\maketitle

\begin{frame}
  \frametitle{Simple example}
  We consider a simple example: computing
  \[
    \alpha = \sum_{i=1}^K \bm v_i^\intercal \bm A \bm v_i
  \]
  where $\bm v \in \mathbb{R}^{N \times K}$ and
  $\bm A \in \mathbb{R}^{N \times N}$. There are $K$ terms, each of which
  require us to compute a matrix-vector product and an inner product.
\end{frame}

\begin{frame}[fragile]
  \frametitle{Serial}
\begin{lstlisting}[style=c, basicstyle=\ttfamily\footnotesize]
void MxV(double *u, double **A, double *v, int N)
{
  for (size_t i = 0; i < N; i++) {
    u[i] = 0.0;
    for (size_t j = 0; j < N; j++)
      u[i] += A[i][j] * v[j];
  }
}

double innerproduct(double *u, double *v, int N)
{
  double result = 0.0;
  for (size_t i = 0; i < N; i++)
    result += u[i] * v[i];
  return result;
}
\end{lstlisting}
\end{frame}

\begin{frame}[fragile]
  \frametitle{Serial}
\begin{lstlisting}[style=c]
double dosum(double **A, double **v, int K, int N)
{
  double alpha = 0.0, temp[N];
  for (size_t i = 0; i < K; i++) {
    MxV(temp, A, v[i], N);
    alpha += innerproduct(temp, v[i], N);
  }

  return alpha;
}
\end{lstlisting}
\end{frame}

\begin{frame}
  \frametitle{OpenMP micro-version}
  \begin{itemize}
  \item It is tempting to exploit all parallelism in sight. However, don't do
    that.
  \item Let us use OpenMP for micro-parallelism. That is, we exploit parallelism
    within the inner product and the matrix-vector operation.
  \item That means two fork/join operations per term, so $2K$ in total.
  \end{itemize}
\end{frame}

\begin{frame}[fragile]
  \frametitle{OpenMP micro-version}
\begin{lstlisting}[style=c]
void MxV(double *u, double **A, double *v, int N)
{
  #pragma omp parallel for schedule(static)
  for (size_t i = 0; i < N; i++) {
    u[i] = 0.0;
    for (size_t j = 0; j < N; j++)
      u[i] += A[i][j] * v[j];
  }
}
\end{lstlisting}
\end{frame}

\begin{frame}[fragile]
  \frametitle{OpenMP micro-version}
\begin{lstlisting}[style=c]
double innerproduct(double *u, double *v, int N)
{
  double result = 0.0;
  #pragma omp parallel for schedule(static) \
          reduction(+:result)
  for (size_t i = 0; i < N; i++)
    result += u[i] * v[i];
  return result;
}
\end{lstlisting}
\end{frame}

\begin{frame}
  \frametitle{OpenMP macro-version}
  \begin{itemize}
  \item The alternative is to exploit the coarsest parallelism: each iteration
    in the \texttt{dosum} method.
  \item In this case we perform exactly one fork and one join.
  \item Problem: the \texttt{dosum} method uses a temporary buffer for the
    matrix-vector multiplication, which cannot be shared between threads. We
    have to use a separate buffer for each thread.
  \end{itemize}
\end{frame}

\begin{frame}[fragile]
  \frametitle{OpenMP macro-version}
\begin{lstlisting}[style=c]
double dosum(double **A, double **v, int K, int N)
{
  double alpha = 0.0;
  double **temp = createMatrix(K, N);
  #pragma omp parallel for schedule(static) \
          reduction(+:alpha)
  for (size_t i = 0; i < K; i++) {
    MxV(temp[i], A, v[i], N);
    alpha += innerproduct(temp[i], v[i], N);
  }

  return alpha;
}
\end{lstlisting}
\end{frame}

\begin{frame}[fragile]
  \frametitle{MPI macro-version}
\begin{lstlisting}[style=c]
double dosumMPI(double **A, double **v,
                int myK, int N)
{
  double myalpha = dosum(A, V, myK, N);
  double alpha;
  MPI_Allreduce(&myalpha, &alpha, 1, MPI_DOUBLE,
                MPI_SUM, MPI_COMM_WORLD);
  return alpha;
}
\end{lstlisting}
  In addition to the usual MPI code, you have to decide on a particular division
  of the work among the nodes. For brevity's sake, it is left out in this
  example.
\end{frame}

\begin{frame}
  \frametitle{Speedup results}
  \begin{center}
    \begin{table}
      \caption{$N=2048$, $K=1024$}
      \bgroup\def\arraystretch{1.2}
      \begin{tabular}{rrrr}
        \hline
        Threads & Micro & Macro & MPI \\
        \hhline{====}
        1 & 1.00 & 1.00 & 1.00 \\
        2 & 1.84 & 1.83 & 1.56 \\
        4 & 2.79 & 2.76 & 3.46 \\
        \hline
      \end{tabular}
      \egroup
    \end{table}
    \begin{table}
      \caption{$N=16$, $K=32768$}
      \bgroup\def\arraystretch{1.2}
      \begin{tabular}{rrrr}
        \hline
        Threads & Micro & Macro & MPI \\
        \hhline{====}
        1 & 1.00 & 1.00 & 1.00 \\
        2 & 0.50 & 2.00 & 2.00 \\
        4 & 0.33 & 3.49 & 4.00 \\
        \hline
      \end{tabular}
      \egroup
    \end{table}
  \end{center}
\end{frame}

\begin{frame}
  \frametitle{BLAS}
  \begin{itemize}
  \item Can the serial code do better? Let's try.
  \item A very helpful library here is \emph{BLAS}: Basic Linear Algebra
    Subprograms.
  \item BLAS is an old standard, from the late seventies and early eighties.
  \item It consist of a bunch of functions with strangely cryptic and short
    names.
  \item Can be installed on Ubuntu with
    \texttt{sudo apt-get install libblas-dev}.
  \item On Vilje, Intel's implementation of BLAS is available under \emph{MKL}:
    the Math Kernel Library.
  \end{itemize}
\end{frame}

\begin{frame}
  \frametitle{BLAS conventions}
  \begin{itemize}
  \item All functions in BLAS starts with one of the letters
    \begin{itemize}
    \item[s] for \emph{single}.
    \item[d] for \emph{double}.
    \item[c] for \emph{single complex}.
    \item[z] for \emph{double complex}.
    \end{itemize}
  \item If the operation involves a matrix, two letters describing the matrix
    format follow. The most important of these are
    \begin{itemize}
    \item[ge] for \emph{general} matrices.
    \item[po] for \emph{symmetric} matrices.
    \item[gb] for \emph{general banded} matrices.
    \item[pb] for \emph{symmetric banded} matrices.
    \end{itemize}
  \end{itemize}
\end{frame}

\begin{frame}
  \frametitle{BLAS levels}
  \begin{itemize}
  \item BLAS functions are organized by \emph{level}.
  \item Level 1: vector-vector operations.
    \[ \bm y = \alpha \bm x + \bm y \]
    \texttt{daxpy(n, alpha, y, 1, x, 1)}
  \item Level 2: matrix-vector operations.
    \[ \bm y = \alpha \bm A \bm x + \beta \bm y \]
    \texttt{dgemv('N', m, n, alpha, A, m, x, 1, beta, y, 1)}
  \item Level 3: matrix-matrix operations
    \[ \bm C = \alpha \bm A \bm B + \beta \bm C \]
    \texttt{dgemm('N', 'N', m, n, k, alpha, A, m, B, k, C, m)}
  \end{itemize}
\end{frame}

\begin{frame}
  \frametitle{BLAS}
  \begin{itemize}
  \item The BLAS home page can be found at http://netlib.org/blas/
  \item BLAS is written in Fortran and therefore expects Fortran memory layout
    (column-major ordering).
  \item For C, the \emph{CBLAS} implementation is popular. CBLAS supports both
    row- and column-major orders.
  \end{itemize}
\end{frame}

\begin{frame}[fragile]
  \frametitle{Serial BLAS}
\begin{lstlisting}[style=c, basicstyle=\ttfamily\footnotesize]
void MxV(double *u, double *A, double *v, int N)
{
  dgemv('N', N, N, 1.0, A, N, v, 1, 0.0, u, 1);
}

double innerproduct(double *u, double *v, int N)
{
  // Necessary adjustments must be made for MPI
  return ddot(N, u, 1, v, 1);
}
\end{lstlisting}
\end{frame}

\begin{frame}
  \frametitle{Timing results}
  \begin{center}
    \begin{table}
      \caption{$N=2048$, $K=1024$}
      \bgroup\def\arraystretch{1.2}
      \begin{tabular}{rrrrr}
        \hline
        Threads & Macro & w/ BLAS & MPI & w/ BLAS \\
        \hhline{=====}
        1 & 35.20 & 2.06 & 35.27 & 2.05 \\
        2 & 17.68 & 1.06 & 18.73 & 1.17 \\
        4 &  9.08 & 0.66 &  9.15 & 0.61 \\
        8 &  4.54 & 0.36 &  4.82 & 0.32 \\
        \hline
      \end{tabular}
      \egroup
    \end{table}
  \end{center}
\end{frame}

\begin{frame}
  \frametitle{Timing results}
  \begin{center}
    \begin{table}
      \caption{$N=16$, $K=32768$ (milliseconds)}
      \bgroup\def\arraystretch{1.2}
      \begin{tabular}{rrrrr}
        \hline
        Threads & Macro & w/ BLAS & MPI & w/ BLAS \\
        \hhline{=====}
        1 &  9.44 &  9.10 & 10.71 & 9.36 \\
        2 & 20.08 & 24.31 &  7.62 & 4.48 \\
        4 & 15.20 & 28.78 &  6.20 & 6.23 \\
        8 &  7.36 & 23.89 &  5.58 & 4.68 \\
        \hline
      \end{tabular}
      \egroup
    \end{table}
  \end{center}
\end{frame}

\input{postamble}

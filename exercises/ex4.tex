\documentclass[onecolumn, oneside, a4paper, 11pt]{memoir}

\usepackage[utf8]{inputenc}
\usepackage[T1]{fontenc}

% Paths
\newcommand{\figs}{../figs}
\newcommand{\data}{../data}

% Fonts
\usepackage{newpxtext,newpxmath}
\renewcommand*\sfdefault{cmss}

% References
\usepackage{hyperref}

% Units
\usepackage[detect-weight=true, binary-units=true]{siunitx}
\DeclareSIUnit\flop{Flops}

% Math
\let\openbox\undefined
\usepackage{amsthm}
\usepackage{amsmath}
\usepackage{amssymb}
\usepackage{bm}

\theoremstyle{remark}
\newtheorem{ex}{Exercise}
\newtheorem*{sol}{Solution}

% Graphics
\usepackage{graphicx}
\usepackage{caption}
\usepackage{subcaption}
\graphicspath{{../figs/}}

% Tikz
\usepackage{tikz}
\usetikzlibrary{positioning,shapes,arrows,calc,intersections}
\usepackage{pgfplots}
\usepgfplotslibrary{dateplot}
\pgfplotsset{compat=1.8}

% Colors
\definecolor{darkblue}{HTML}{00688B}
\definecolor{darkgreen}{HTML}{6E8B3D}
\definecolor{cadet}{HTML}{DAE1FF}
\definecolor{salmon}{HTML}{FFB08A}

% Listings
\usepackage{textcomp}
\usepackage{listings}
\lstset{
  keywordstyle=\bfseries\color{orange},
  stringstyle=\color{darkblue!80},
  commentstyle=\color{darkblue!80},
  showstringspaces=false,
  basicstyle=\ttfamily,
  upquote=true,
}
\lstdefinestyle{fortran}{
  language=Fortran,
  morekeywords={for},
  deletekeywords={status},
}
\lstdefinestyle{c}{
  language=C,
  morekeywords={include},
}
\lstdefinestyle{shell}{
  language=bash,
}

\begin{document}

\pagestyle{empty}

\begin{center}
  {\Huge \bfseries \scshape
    Introduction to \\[0.2\baselineskip] Supercomputing} \\[2\baselineskip]
  {\Large TMA4280 $\cdot$ Problem set 4} \\[2\baselineskip]
\end{center}

\textbf{Note that}
\begin{itemize}
\item this problem set is mandatory;
\item you can work on it in groups with \emph{up to} 3 members;
\item you should write a report describing your solution, as continuous text;
\item the source code should be handed in together with the report (and not as
  part of it);
\item please make sure that you have answered all the questions;
\item the due date is Friday, February 19, 2015;
\item the report will count 10\% towards the final grade.
\end{itemize}

We consider a vector $\bm v \in \mathbb{R}^n$ where the vector elements are
defined as
\[
  v_i = \frac{1}{i^2}, \qquad i=1,\ldots,n.
\]
We are interested in computing the sum of all vector elements numerically, i.e.
we want to find $S_n$, where
\[
  S_n = \sum_{i=1}^n v_i.
\]
Note that the limit is
\[
  S = \lim_{n\to\infty} S_n = \frac{\pi^2}{6}.
\]

\begin{ex}
  Write a program, in either C or Fortran, that
  \begin{itemize}
  \item generates the vector $\bm v$;
  \item computes the sum $S_n$ in double precision on a single process;
  \item computes the error $|S - S_n|$ for $n=2^k$ with $k=3,4,\ldots,14$;
  \item prints out the error $|S - S_n|$ in double precision.
  \end{itemize}
\end{ex}

\begin{ex}
  Make the necessary changes needed to use shared memory parallelization with
  OpenMP.
\end{ex}

\begin{ex}
  Write a program that computes the sum $S_n$ using a distributed memory model
  (MPI) on $P$ processors, where $P$ is a power of two.

  The program should work as follows. Only process zero (the root) should be
  responsible for generating the vector elements. The root should partition and
  distribute the elements evenly among all the processes. Each process should
  sum up its own part. At the end, all the partial sums should be added together
  and made available on the root for printing. Report the error $|S-S_n|$ in
  double precision for different values of $n$.
\end{ex}

\begin{ex}
  Confirm that your program also works when using OpenMP and MPI in
  combination.
\end{ex}

\begin{ex}
  Which MPI calls were convenient and/or necessary to use?
\end{ex}

\begin{ex}
  Compare the errors from the single-process program and the multi-process
  program for $P=2$ and $P=8$. Should the answer be the same in all cases?
  Exactly, or approximately?
\end{ex}

\begin{ex}
  Compare the memory requirement per process for the single-process program and
  the multi-process program when $n \gg 1$.
\end{ex}

\begin{ex}
  How many floating point operations are needed to generate the vector $\bm v$?
  How many are needed to compute $S_n$? Is the multi-process program load
  balanced?
\end{ex}

\begin{ex}
  Do you consider parallel processing attractive for solving this problem?
\end{ex}

\end{document}

\documentclass[onecolumn, oneside, a4paper, 11pt]{memoir}

\usepackage[utf8]{inputenc}
\usepackage[T1]{fontenc}

% Paths
\newcommand{\figs}{../figs}
\newcommand{\data}{../data}

% Fonts
\usepackage{newpxtext,newpxmath}
\renewcommand*\sfdefault{cmss}

% Math
\usepackage{amsmath}
\usepackage{amssymb}
\usepackage{bm}

\newtheorem{ex}{Exercise}

\begin{document}

\pagestyle{empty}

\begin{center}
  {\Huge \bfseries \scshape
    Introduction to \\[0.2\baselineskip] Supercomputing} \\[2\baselineskip]
  {\Large TMA4280 $\cdot$ Problem set 1} \\[2\baselineskip]
\end{center}

\begin{ex}
  Exercise 1.1 in the lecture notes.
\end{ex}

\begin{ex}
  Exercise 1.2 in the lecture notes.
\end{ex}

\begin{ex}
  Exercise 1.3 in the lecture notes.
\end{ex}

\begin{ex}
  Exercise 1.4 in the lecture notes.
\end{ex}

\begin{ex}
  Exercise 1.5 in the lecture notes.
\end{ex}

\begin{ex}
  Exercise 1.6 in the lecture notes.
\end{ex}

\begin{ex}
  In the lecture we found that adding a small number to a large number can cause
  problems when the relative difference between the numbers exceed the accuracy
  of the floating point representation.

  With this in mind, suggest an algorithm for summing a list of numbers that is
  more accurate than doing it ``naively''.
\end{ex}

\begin{ex}
  Implement a C or Fortran program that calculates $\bm y = \bm A \bm x$.
  \begin{align*}
    \bm A =
    \begin{pmatrix}
      0.3 & 0.4 & 0.3 \\
      0.7 & 0.1 & 0.2 \\
      0.5 & 0.5 & 0.0
    \end{pmatrix}, \qquad \bm x = \begin{pmatrix} 1.0 \\ 1.0 \\ 1.0 \end{pmatrix}.
  \end{align*}
\end{ex}

\end{document}

\documentclass[onecolumn, oneside, a4paper, 11pt]{memoir}

\usepackage[utf8]{inputenc}
\usepackage[T1]{fontenc}

% Paths
\newcommand{\figs}{../figs}
\newcommand{\data}{../data}

% Fonts
\usepackage{newpxtext,newpxmath}
\renewcommand*\sfdefault{cmss}

% References
\usepackage{hyperref}

% Units
\usepackage[detect-weight=true, binary-units=true]{siunitx}
\DeclareSIUnit\flop{Flops}

% Math
\let\openbox\undefined
\usepackage{amsthm}
\usepackage{amsmath}
\usepackage{amssymb}
\usepackage{bm}

\theoremstyle{remark}
\newtheorem{ex}{Exercise}
\newtheorem*{sol}{Solution}

% Graphics
\usepackage{graphicx}
\usepackage{caption}
\usepackage{subcaption}
\graphicspath{{../figs/}}

% Tikz
\usepackage{tikz}
\usetikzlibrary{positioning,shapes,arrows,calc,intersections}
\usepackage{pgfplots}
\usepgfplotslibrary{dateplot}
\pgfplotsset{compat=1.8}

% Colors
\definecolor{darkblue}{HTML}{00688B}
\definecolor{darkgreen}{HTML}{6E8B3D}
\definecolor{cadet}{HTML}{DAE1FF}
\definecolor{salmon}{HTML}{FFB08A}

% Listings
\usepackage{textcomp}
\usepackage{listings}
\lstset{
  keywordstyle=\bfseries\color{orange},
  stringstyle=\color{darkblue!80},
  commentstyle=\color{darkblue!80},
  showstringspaces=false,
  basicstyle=\ttfamily,
  upquote=true,
}
\lstdefinestyle{fortran}{
  language=Fortran,
  morekeywords={for},
  deletekeywords={status},
}
\lstdefinestyle{c}{
  language=C,
  morekeywords={include},
}
\lstdefinestyle{shell}{
  language=bash,
}

\begin{document}

\pagestyle{empty}

\begin{center}
  {\Huge \bfseries \scshape
    Introduction to \\[0.2\baselineskip] Supercomputing} \\[2\baselineskip]
  {\Large TMA4280 $\cdot$ Problem set 5} \\[2\baselineskip]
\end{center}

\begin{ex} Answer the following questions.
  \begin{enumerate}
  \item On which multi-processor systems is it of interest to use message
    passing?
  \item What are the advantages of using a standardized communication library
    (or message passing library) such as MPI?
  \item A communication library consists of many specific message passing
    operations. How would you classify these operations into a few main groups,
    or types of communication patterns?
  \item Explain what is wrong with the following code segment. It is written in
    C but the language is not important.
    \begin{lstlisting}[style=c]
  MPI_Comm_rank(comm, &rank);
  if (rank == 0) {
    MPI_Recv(recvbuf, count, MPI_DOUBLE, 1,
             tag, comm, &status);
    MPI_Send(sendbuf, count, MPI_DOUBLE, 1, tag, comm);
  }
  else if (rank == 1) {
    MPI_Recv(recvbuf, count, MPI_DOUBLE, 0,
             tag, comm, &status);
    MPI_Send(sendbuf, count, MPI_DOUBLE, 0, tag, comm);
  }
    \end{lstlisting}
  \end{enumerate}
\end{ex}

\begin{ex}
  Assume a distributed memory multiprocessor computer with the following
  interconnect charateristic: the time $\tau(k)$ it takes to send a message with
  $k$ bytes can be approximated as
  \[
    \tau(k) = \tau_\text{s} + \beta k,
  \]
  where $\tau_\text{s}$ is a fixed startup time and $\beta$ is the inverse
  bandwidth (units of seconds per byte).

  For our setup, $\tau_\text{s} = \SI{1}{\micro\second}$ and
  $\beta = \SI{1.25}{\nano\second\per\byte}$.
  \begin{enumerate}
  \item How many bytes can we send in a single message before the time to send
    the message is twice the startup time? To how many (double precision)
    floating-point numbers does this correspond?
  \item How long does it take to send a message with a single floating point
    number? Is it preferable to send many short messages instead of a single,
    long one?
  \end{enumerate}
\end{ex}

\begin{ex}
  Let $\bm A, \bm B, \bm C, \bm D$ be matrices of size $n \times n$, and let the
  matrix $\bm D$ be constructed as
  \[
    \bm D = 3 \bm A \bm B + \bm C.
  \]
  How many floating point operations does it take to construct $\bm D$?
\end{ex}

\begin{ex}
  Implement an MPI-based matrix multiplication program. It should accept one
  command-line argument, which is the size of the matrix and vector to multiply.
  Use arbitrary data (e.g. random).

  You are free to choose the structure of the matrix and vector, but a typical
  approach is the following:
  \begin{itemize}
  \item Each process ``owns'' a certain set of indices.
  \item Each process only stores the part of the vector that it owns, and the
    part of the matrix corresponding to the \emph{columns} that it owns.
  \item The result of the local matrix-vector product is then a full-sized
    vector, with contributions to the vector chunks of all other processes.
  \end{itemize}

  Things to consider:
  \begin{enumerate}
  \item What changes will you have to make if you want to store the matrix on each
    process by rows instead of columns?
  \item For some types of matrices it is possible to minimize communication by
    cleverly choosing the index sets. What characterizes these matrices?
  \item What would a matrix transpose operation look like?
  \end{enumerate}
\end{ex}

\end{document}

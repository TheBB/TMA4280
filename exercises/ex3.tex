\documentclass[onecolumn, oneside, a4paper, 11pt]{memoir}

\usepackage[utf8]{inputenc}
\usepackage[T1]{fontenc}

% Paths
\newcommand{\figs}{../figs}
\newcommand{\data}{../data}

% Fonts
\usepackage{newpxtext,newpxmath}
\renewcommand*\sfdefault{cmss}

% References
\usepackage{hyperref}

% Units
\usepackage[detect-weight=true, binary-units=true]{siunitx}
\DeclareSIUnit\flop{Flops}

% Math
\let\openbox\undefined
\usepackage{amsthm}
\usepackage{amsmath}
\usepackage{amssymb}
\usepackage{bm}

\theoremstyle{remark}
\newtheorem{ex}{Exercise}
\newtheorem*{sol}{Solution}

% Graphics
\usepackage{graphicx}
\usepackage{caption}
\usepackage{subcaption}
\graphicspath{{../figs/}}

% Tikz
\usepackage{tikz}
\usetikzlibrary{positioning,shapes,arrows,calc,intersections}
\usepackage{pgfplots}
\usepgfplotslibrary{dateplot}
\pgfplotsset{compat=1.8}

% Colors
\definecolor{darkblue}{HTML}{00688B}
\definecolor{darkgreen}{HTML}{6E8B3D}
\definecolor{cadet}{HTML}{DAE1FF}
\definecolor{salmon}{HTML}{FFB08A}

% Listings
\usepackage{textcomp}
\usepackage{listings}
\lstset{
  keywordstyle=\bfseries\color{orange},
  stringstyle=\color{darkblue!80},
  commentstyle=\color{darkblue!80},
  showstringspaces=false,
  basicstyle=\ttfamily,
  upquote=true,
}
\lstdefinestyle{fortran}{
  language=Fortran,
  morekeywords={for},
  deletekeywords={status},
}
\lstdefinestyle{c}{
  language=C,
  morekeywords={include},
}
\lstdefinestyle{shell}{
  language=bash,
}

\begin{document}

\pagestyle{empty}

\begin{center}
  {\Huge \bfseries \scshape
    Introduction to \\[0.2\baselineskip] Supercomputing} \\[2\baselineskip]
  {\Large TMA4280 $\cdot$ Problem set 3} \\[2\baselineskip]
\end{center}

\begin{ex}
  Go to \url{http://top500.org}, a website cataloguing the top 500
  supercomputers in the world. Study the top 10, in particular their technical
  specifications. What is meant by the LINPACK benchmark performance?
\end{ex}

\begin{ex}
  Note that some of this was not covered in detail in the lecture. Please have a
  look at the lecture notes.
  \begin{enumerate}
  \item What limits the scalability of a bus-based interconnect?
  \item How are the individual processors connected using a crossbar?
  \item How are the individual processors connected using a mesh?
  \item What is the difference between a shared-memory and a distributed
    memory architecture?
  \item What characterizes the memory access in an SMP?
  \item What is the difference between a NUMA and a ccNUMA architecture?
  \end{enumerate}
\end{ex}

\begin{ex}
  How many bytes are sent in each of the three messages listed below? Here given
  in C, but that's not important.

\begin{lstlisting}[style=c]
MPI_Send(buf1, 80, MPI_CHAR, dest, tag, MPI_COMM_WORLD);
MPI_Send(buf2, 1024, MPI_INT, dest, tag, MPI_COMM_WORLD);
MPI_Send(buf3, 1024, MPI_DOUBLE, dest, tag, MPI_COMM_WORLD);
\end{lstlisting}
\end{ex}

\begin{ex}
  Is it true that a unique tag must be specified each time \texttt{MPI\_Recv} is
  called?
\end{ex}

\begin{ex}
  Implement the program from the previous exercise using MPI. It should run on
  any number of processes.

  Hint: Partition the matrix in column strips.
\end{ex}

\end{document}
